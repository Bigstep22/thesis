\long\def\ignore#1{}
\ignore{
\begin{code}
{-# OPTIONS_GHC -ddump-simpl -ddump-to-file -dsuppress-all -dno-suppress-type-signatures -dno-typeable-binds -dsuppress-uniques #-} {-# OPTIONS_GHC -Wno-missing-signatures #-}
{-# OPTIONS_GHC -Wno-unused-top-binds #-}
{-# OPTIONS_GHC -Wno-name-shadowing #-}
{-# OPTIONS_GHC -Wno-unused-matches #-}
{-# LANGUAGE DeriveFunctor #-}
{-# LANGUAGE RankNTypes #-}
{-# LANGUAGE ExistentialQuantification #-}
import Test.Tasty.Bench
\end{code}
}
\subsection{Leaf Trees}
In this section, I describe my replication of \cite{Harper2011}'s code.
We start with his motivating example at the beginning of the paper, followed by the `fused' version that we want the pipeline to become, once compiled:
\begin{code}
f :: (Int, Int) -> Int
f = sum1 . map1 (+1) . filter1 odd . between1

f' :: (Int, Int) -> Int
f' (x, y) = loop x
  where loop x = if x > y
                 then 0
                 else if odd x
                      then (x + 1) + loop (x + 1)
                      else loop (x + 1)
\end{code}
\paragraph{Datatypes} In his paper Harper implemented his example functions using leaf trees, this is defined as \tt{Tree} below.
Furthermore, the base functor of \tt{Tree} was defined, as \tt{Tree\_}, with the recursive positions of the functor turned into a parameter of the datatype:
\begin{code}
data Tree a = Empty | Leaf a | Fork (Tree a) (Tree a)
data Tree_ a b = Empty_ | Leaf_ a | Fork_ b b
\end{code}
\paragraph{Church encoding} The Church encoding of the \tt{Tree} datatype is defined using the base functor:
\begin{code}
data TreeCh a = TreeCh (forall b . (Tree_ a b -> b) -> b)
\end{code}
Next, the conversion functions \tt{toCh} and \tt{fromCh} are defined, using two auxiliary functions \tt{fold} and \tt{in'}:
\begin{code}
toCh :: Tree a -> TreeCh a
toCh t = TreeCh (\a -> fold a t)
  where fold :: (Tree_ a b -> b) -> Tree a -> b
        fold a Empty      = a Empty_
        fold a (Leaf x)   = a (Leaf_ x)
        fold a (Fork l r) = a (Fork_ (fold a l)
                             (fold a r))
fromCh :: TreeCh a -> Tree a
fromCh (TreeCh fold) = fold in'
  where in' :: Tree_ a (Tree a) -> Tree a
        in' Empty_ = Empty
        in' (Leaf_ x) = Leaf x
        in' (Fork_ l r) = Fork l r
\end{code}
From here, the fusion rule is defined using a \tt{RULES} pragma. Along with a couple of other rules, this core construct is responsible for doing the actual `fusion'.
The \tt{INLINE} pragmas are also included, to delay any inlining of the \tt{toCh/fromCh} functions to the latest possible moment, maximizing the opportunity for fusion throughout the compilation process:
\begin{code}
{-# RULES "toCh/fromCh fusion" forall x. toCh (fromCh x) = x #-}
{-# INLINE [0] toCh #-}
{-# INLINE [0] fromCh #-}
\end{code}
A generalized natural transformation function is defined to standardize and ease later implementations of transformation functions:
\begin{code}
natCh :: (forall c . Tree_ a c -> Tree_ b c) -> TreeCh a -> TreeCh b
natCh f (TreeCh g) = TreeCh (\a -> g (a . f))
\end{code}
\paragraph{Cochurch encoding} Conversely, the Cochurch encoding is defined, again using the base functor for \tt{Tree}:
\begin{code}
data TreeCoCh a = forall s . TreeCoCh (s -> Tree_ a s) s
\end{code}

Next, the conversion functions \tt{toCoCh} and \tt{fromCoCh} are again defined, using two auxiliary functions \tt{out} and \tt{unfold}:
\begin{code}
toCoCh :: Tree a -> TreeCoCh a
toCoCh = TreeCoCh out
  where out Empty = Empty_
        out (Leaf a) = Leaf_ a
        out (Fork l r) = Fork_ l r
fromCoCh :: TreeCoCh a -> Tree a
fromCoCh (TreeCoCh h s) = unfold h s
  where unfold h s = case h s of
          Empty_ -> Empty
          Leaf_ a -> Leaf a
          Fork_ sl sr -> Fork (unfold h sl) (unfold h sr)
\end{code}
Similar to Church encodings, the proper pragmas are included to enable fusion:
\begin{code}
{-# RULES "toCh/fromCh fusion" forall x. toCoCh (fromCoCh x) = x #-}
{-# INLINE [0] toCoCh #-}
{-# INLINE [0] fromCoCh #-}
\end{code}
A generalized natural transformation function is defined to standardize and ease later implementations of transformation functions:
\begin{code}
natCoCh :: (forall c . Tree_ a c -> Tree_ b c) -> TreeCoCh a -> TreeCoCh b
natCoCh f (TreeCoCh h s) = TreeCoCh (f . h) s
\end{code}
\paragraph{Between} Three between functions are implemented:
One regular, one Church encoded, and one Cochurch encoded.
Note how all three final functions are accompanied by an \tt{INLINE} pragma. This inlining enables pairs of \tt{toCh $\circ$ fromCh} to be revealed to the compiler for fusion.
The non-encoded function is implemented recursively in a fashion appropriate for leaf trees:
\begin{code}
between1 :: (Int, Int) -> Tree Int
between1 (x, y) = case compare x y of
  GT -> Empty
  EQ -> Leaf x
  LT -> Fork (between1 (x, mid))
               (between1 (mid + 1, y))
  where mid = (x + y) `div` 2
\end{code}
The Church encoded version leverages the implementation of a recursion principle \tt{b} for the between function of leaf trees:
\begin{code}
between2 :: (Int, Int) -> Tree Int
between2 = fromCh . betweenCh
  where betweenCh :: (Int, Int) -> TreeCh Int
        betweenCh (x, y) = TreeCh (\a -> b a (x, y))
        b :: (Tree_ Int b -> b) -> (Int, Int) -> b
        b a (x, y) = case compare x y of
          GT -> a Empty_
          EQ -> a (Leaf_ x)
          LT -> a (Fork_ (b a (x, mid))
                         (b a (mid + 1, y)))
          where mid = (x + y) `div` 2
{-# INLINE between2 #-}
\end{code}
The Cochurch encoded version leverages the implementation of a coalgebra \tt{h} for the between function of leaf trees:
\begin{code}
between3 :: (Int, Int) -> Tree Int
between3 = fromCoCh . TreeCoCh h
  where h :: (Int, Int) -> Tree_ Int (Int, Int)
        h (x, y) = case compare x y of
          GT -> Empty_
          EQ -> Leaf_ x
          LT -> Fork_ (x, mid) (mid + 1, y)
          where mid = (x + y) `div` 2
{-# INLINE between3 #-}
\end{code}
\ignore{
\begin{code}
append1 :: Tree a -> Tree a -> Tree a
append1 t1 Empty = t1
append1 Empty t2 = t2
append1 t1 t2 = Fork t1 t2

appendCh :: TreeCh a -> TreeCh a -> TreeCh a
appendCh (TreeCh g1) (TreeCh g2) =
    TreeCh (\a -> a (Fork_ (g1 a) (g2 a)))
append2 :: Tree a -> Tree a -> Tree a
append2 t1 t2 = fromCh (appendCh (toCh t1) (toCh t2))
{-# INLINE append2 #-}

appendCoCh :: TreeCoCh a -> TreeCoCh a -> TreeCoCh a
appendCoCh (TreeCoCh h1 s1) (TreeCoCh h2 s2) = TreeCoCh h' Nothing
  where h' Nothing = Fork_ (Just (TreeCoCh h1 s1))
                           (Just (TreeCoCh h2 s2))
        h' (Just (TreeCoCh h s)) = case h s of
          Empty_ -> Empty_
          Leaf_ a -> Leaf_ a
          Fork_ l r -> Fork_ (Just (TreeCoCh h l))
                             (Just (TreeCoCh h r))
append3 :: Tree a -> Tree a -> Tree a
append3 t1 t2 = fromCoCh (appendCoCh (toCoCh t1) (toCoCh t2))
{-# INLINE append3 #-}
\end{code}
{-# RULES
"append -> fused" [~1] forall t1 t2.
  append1 t1 t2 =
    fromCh (appendCh (toCh t1) (toCh t2))
"append -> unfused" [1] forall t1 t2.
  fromCh (appendCh (toCh t1) (toCh t2)) =
    append1 t1 t2 #-}
-- This offers 10% speedup in the scenario when a fused
-- pipeline has been built, but fusion has not been enabled
-- 393 +- 14 micro seconds vs.
-- 434 +- 42 micro seconds.
}
\paragraph{Filter} Again three versions, similar to between.
The regular implementation is as to be expected, leveraging an implementation of append:
\begin{code}
filter1 :: (a -> Bool) -> Tree a -> Tree a
filter1 p Empty = Empty
filter1 p (Leaf a) = if p a then Leaf a else Empty
filter1 p (Fork l r) = append1 (filter1 p l) (filter1 p r)
\end{code}

While for the (Co)Church encoded versions, a natural transformation \tt{filt} is constructed.
This is used to both implement both the Church and Cochurch encoded function:
\begin{code}
filt :: (a -> Bool) -> Tree_ a c -> Tree_ a c
filt p Empty_ = Empty_
filt p (Leaf_ x) = if p x then Leaf_ x else Empty_
filt p (Fork_ l r) = Fork_ l r
filter2 :: (a -> Bool) -> Tree a -> Tree a
filter2 p = fromCh . natCh (filt p) . toCh
{-# INLINE filter2 #-}
filter3 :: (a -> Bool) -> Tree a -> Tree a
filter3 p = fromCoCh . natCoCh (filt p) . toCoCh
{-# INLINE filter3 #-}
\end{code}
\paragraph{Map} The map function is implemented similarly to filter: A simple implementation for the non-encoded version and a single natural transformation that is leveraged in both the Church and Cochurch encoded versions:
\begin{code}
map1 :: (a -> b) -> Tree a -> Tree b
map1 f Empty = Empty
map1 f (Leaf a) = Leaf (f a)
map1 f (Fork l r) = append1 (map1 f l) (map1 f r)
m :: (a -> b) -> Tree_ a c -> Tree_ b c
m f Empty_ = Empty_
m f (Leaf_ a) = Leaf_ (f a)
m f (Fork_ l r) = Fork_ l r
map2 :: (a -> b) -> Tree a -> Tree b
map2 f = fromCh . natCh (m f) . toCh
{-# INLINE map2 #-}
map3 :: (a -> b) -> Tree a -> Tree b
map3 f = fromCoCh . natCoCh (m f) . toCoCh
{-# INLINE map3 #-}
\end{code}
\paragraph{Sum} The sum function is again more interesting, it is again implemented in three different ways:
The non-encoded version is again as would normally be expected for leaf trees:
\begin{code}
sum1 :: Tree Int -> Int
sum1 Empty = 0
sum1 (Leaf x) = x
sum1 (Fork x y) = sum1 x + sum1 y
\end{code}
The Church encoded version leverages an algebra \tt{s}:
\begin{code}
sum2 :: Tree Int -> Int
sum2 = sumCh . toCh
  where sumCh :: TreeCh Int -> Int
        sumCh (TreeCh g) = g s
        s :: Tree_ Int Int -> Int
        s Empty_ = 0
        s (Leaf_ x) = x
        s (Fork_ x y) = x + y
{-# INLINE sum2 #-}
\end{code}
The Cochurch encoding is defined using a coinduction principle.
Note that it is possible to implement this function using an accumulator of a list datatype (used like a queue), but it currently does not seem to provide a fused Core AST, for a more expansive discussion on tail-recursive Cochurch encoded pipelines, see \autoref{sec:tail}:
\begin{code}
sum3 :: Tree Int -> Int
sum3 = sumCoCh . toCoCh
  where sumCoCh :: TreeCoCh Int -> Int
        sumCoCh (TreeCoCh h s') = loop s'
          where loop s = case h s of
                  Empty_ -> 0
                  Leaf_ x -> x
                  Fork_ l r -> loop l + loop r
{-# INLINE sum3 #-}
\end{code}
\paragraph{Pipelines} Finally, an example pipeline, whose performance can be measure or Core representation inspected, is defined:
\begin{code}
pipeline1 :: (Int, Int) -> Int
pipeline1 = sum1 . map1 (+2) . filter1 odd . between1
\end{code}
\ignore{
\begin{code}
pipeline2 = sum2 . map2 (+2) . filter2 odd . between2
pipeline3 = sum3 . map3 (+2) . filter3 odd . between3
input = (1, 10000)
--main = print (pipeline3 input)
sumApp1 (x, y)  = sum1 (append1 (between1 (x, y)) (between1 (x, y)))
sumApp2 (x, y)  = sum2 (append2 (between2 (x, y)) (between2 (x, y)))
sumApp3 (x, y)  = sum3 (append3 (between3 (x, y)) (between3 (x, y)))

main = defaultMain
  [
    bgroup "Filter pipeline"
    [ bench "pipcofused1" $ nf pipeline3 input
    , bench "pipcofused2" $ nf pipeline3 input
    , bench "pipcofused3" $ nf pipeline3 input
    , bench "pipcofused4" $ nf pipeline3 input
    , bench "pipchfused1" $ nf pipeline2 input
    , bench "pipchfused2" $ nf pipeline2 input
    , bench "pipchfused3" $ nf pipeline2 input
    , bench "pipchfused4" $ nf pipeline2 input
    , bench "pipunfused1" $ nf pipeline1 input
    , bench "pipunfused2" $ nf pipeline1 input
    , bench "pipunfused3" $ nf pipeline1 input
    , bench "pipunfused4" $ nf pipeline1 input
    ]
    {- ,
    bgroup "Sum-append pipeline"
    [
       bench "sumcofused1" $ nf sumApp3 input             
    ,  bench "sumcofused2" $ nf sumApp3 input             
    ,  bench "sumcofused3" $ nf sumApp3 input             
    ,  bench "sumcofused4" $ nf sumApp3 input             
    ,  bench "sumchfused1" $ nf sumApp2 input
    ,  bench "sumchfused2" $ nf sumApp2 input
    ,  bench "sumchfused3" $ nf sumApp2 input
    ,  bench "sumchfused4" $ nf sumApp2 input
    ,  bench "sumunfused1" $ nf sumApp1 input
    ,  bench "sumunfused2" $ nf sumApp1 input
    ,  bench "sumunfused3" $ nf sumApp1 input
    ,  bench "sumunfused4" $ nf sumApp1 input
    ] -}
  ]
-- With tastybench
{- chfused
Rec {
$s$wb :: Int -> Int# -> Int
$s$wb
  = \ (ww :: Int) (ww1 :: Int#) ->
      case ww of wild { I# x ->
      case ># x ww1 of {
        __DEFAULT ->
          case ==# x ww1 of {
            __DEFAULT ->
              case <# x ww1 of {
                __DEFAULT -> case lvl7 of wild1 { };
                1# ->
                  let {
                    mid :: Int#
                    mid = uncheckedIShiftRA# (+# x ww1) 1# } in
                  case $s$wb wild mid of { I# x1 ->
                  case $s$wb (I# (+# mid 1#)) ww1 of { I# y -> I# (+# x1 y) }
                  }
              };
            1# ->
              case remInt# x 2# of {
                __DEFAULT -> I# (+# x 2#);
                0# -> lvl4
              }
          };
        1# -> lvl4
      }
      }
end Rec }
-}
{- cofused
Rec {
$wloop :: Int# -> Int# -> Int#
$wloop
  = \ (ww :: Int#) (ww1 :: Int#) ->
      case ># ww ww1 of {
        __DEFAULT ->
          case ==# ww ww1 of {
            __DEFAULT ->
              case <# ww ww1 of {
                __DEFAULT -> case lvl3 of wild { };
                1# ->
                  let {
                    mid :: Int#
                    mid = uncheckedIShiftRA# (+# ww ww1) 1# } in
                  case $wloop ww mid of ww2 { __DEFAULT ->
                  case $wloop (+# mid 1#) ww1 of ww3 { __DEFAULT -> +# ww2 ww3 }
                  }
              };
            1# ->
              case remInt# ww 2# of {
                __DEFAULT -> +# ww 2#;
                0# -> 0#
              }
          };
        1# -> 0#
      }
end Rec }
-}
\end{code}
}