\long\def\ignore#1{}
\ignore{
\begin{code}
{-# LANGUAGE DeriveFunctor #-}
{-# LANGUAGE RankNTypes #-}
{-# LANGUAGE LambdaCase #-}
{-# LANGUAGE ExistentialQuantification #-}
{-# OPTIONS_GHC -ddump-simpl -ddump-to-file -dsuppress-all -dno-suppress-type-signatures -dno-typeable-binds -dsuppress-uniques #-}
{-# OPTIONS_GHC -Wno-missing-signatures #-}
{-# OPTIONS_GHC -Wno-unused-top-binds #-}
{-# OPTIONS_GHC -Wno-name-shadowing #-}
{-# OPTIONS_GHC -Wno-unused-matches #-}
{-# OPTIONS_GHC -O2 #-}
import Prelude hiding (foldr, concatMap)
import Test.Tasty.Bench
import GHC.Base
\end{code}
}
\subsection{Lists}
In this section we discuss my further replication of \cite{Harper2011}'s work.
We implement some of the functions and pipelines that Harper described, such as \tt{between}, \tt{filter}, and \tt{sum}, but using the List datatype instead of Leaf Trees.
This was done to see how the descriptions in Harper's work generalize and to have a simpler datastructure on which to perform analysis; seeing how and when the fusion works and when it doesn't.

We again start with the datatype descriptions. We use \tt{List'} instead of \tt{List} as there is a namespace collision with GHC's \tt{List} datatype:
\begin{code}
import GHC.List
data List' a = Nil | Cons a (List' a)
data List_ a b = Nil_ | Cons_ a b
\end{code}
\paragraph{Church encodings} We define the Church encoding and proper encoding and decoding functions:
\begin{code}
data ListCh a = ListCh (forall b . (List_ a b -> b) -> b)
toCh :: List' a -> ListCh a
toCh t = ListCh (\a -> fold a t)
  where fold :: (List_ a b -> b) -> List' a -> b
        fold a Nil         = a Nil_
        fold a (Cons x xs) = a (Cons_ x (fold a xs))

fromCh :: ListCh a -> List' a
fromCh (ListCh fold') = fold' in'
  where in' :: List_ a (List' a) -> List' a
        in' Nil_ = Nil
        in' (Cons_ x xs) = Cons x xs
\end{code}
We omit the pragmas defined for \tt{toCh} and \tt{fromCh} as well as the \tt{natCh}, as their definition is identical to the ones defined for Leaf Trees.
% We introduce three important pragmas.
% One is the actual fusion rule, taking two functions and removing them from the compilation process.
% The second and third are to make sure that the \tt{toCh} and \tt{fromCh} functions are inlined as late as possible; maximizing the time that they can be fused during the compilation process:
\ignore{
\begin{code}
{-# RULES "toCh/fromCh fusion" forall x. toCh (fromCh x) = x #-}
{-# INLINE [0] toCh #-}
{-# INLINE [0] fromCh #-}
\end{code}
}
% A generalized natural transformation function is defined to standardize and ease later implementations of transformation functions:
\ignore{
\begin{code}
natCh :: (forall c . List_ a c -> List_ b c) -> ListCh a -> ListCh b
natCh f (ListCh g) = ListCh (\a -> g (a . f))
\end{code}
}
\paragraph{Cochurch encodings} We defined the Cochurch encodings conversely:
\begin{code}
data ListCoCh a = forall s . ListCoCh (s -> List_ a s) s
toCoCh :: List' a -> ListCoCh a
toCoCh = ListCoCh out
  where out :: List' a -> List_ a (List' a)
        out Nil = Nil_
        out (Cons x xs) = Cons_ x xs

fromCoCh :: ListCoCh a -> List' a
fromCoCh (ListCoCh h s) = unfold h s
  where unfold :: (b -> List_ a b) -> b -> List' a
        unfold h s = case h s of
          Nil_ -> Nil
          Cons_ x xs -> Cons x (unfold h xs)
\end{code}
\ignore{
\begin{code}
{-# RULES "toCh/fromCh fusion" forall x. toCoCh (fromCoCh x) = x #-}
{-# INLINE [0] toCoCh #-}
{-# INLINE [0] fromCoCh #-}
\end{code}
}
% A generalized natural transformation function is defined to standardize and ease later implementations of transformation functions:
\ignore{
\begin{code}
natCoCh :: (forall c . List_ a c -> List_ b c) -> ListCoCh a -> ListCoCh b
natCoCh f (ListCoCh h s) = ListCoCh (f . h) s
\end{code}
}
\paragraph{Between}
The between function is defined in three different fashions: Normally, with the Church encoding, and with the Cochurch encoding.
We leverage \tt{INLINE} pragmas to make sure that the fusion pragmas can effectively work.
For the non-encoded implementation, we simply leverage recursion:
\begin{code}
between1 :: (Int, Int) -> List' Int
between1 (x, y) = case x > y of
  True  -> Nil
  False -> Cons x (between1 (x+1,y))
{-# INLINE between1 #-}
\end{code}
For the Church encoded version we define a recursion principle \tt{b} and use that to define the encoded Church function:
\begin{code}
between2 :: (Int, Int) -> List' Int
between2 = fromCh . betweenCh
  where betweenCh :: (Int, Int) -> ListCh Int
        betweenCh (x, y) = ListCh (\a -> b a (x, y))
        b :: (List_ Int b -> b) -> (Int, Int) -> b
        b a (x, y) = loop x
          where loop x = if x > y
                         then a Nil_
                         else a (Cons_ x (loop (x+1)))
        {-# INLINE betweenCh #-}
{-# INLINE between2 #-}
\end{code}
For the Cochurch encoded version we define a coalgebra:
\begin{code}
between3 :: (Int, Int) -> List' Int
between3 = fromCoCh . ListCoCh betweenCoCh
  where betweenCoCh :: (Int, Int) -> List_ Int (Int, Int)
        betweenCoCh (x, y) = if x > y 
                             then Nil_
                             else Cons_ x (x+1, y)
        {-# INLINE betweenCoCh #-}
{-# INLINE between3 #-}
\end{code}

\paragraph{Map}
It is possible to implement the \tt{map} function using a natural transformation. Again three implementations, the latter two of which leverage the defined natural transformation \tt{m}:
\begin{code}
map1 :: (a -> b) -> List' a -> List' b
map1 _ Nil = Nil
map1 f (Cons x xs) = Cons (f x) (map1 f xs)
{-# INLINE map1 #-}
m :: (a -> b) -> List_ a c -> List_ b c
m f (Cons_ x xs) = Cons_ (f x) xs
m _ Nil_ = Nil_

map2 :: (a -> b) -> List' a -> List' b
map2 f = fromCh . natCh (m f) . toCh
{-# INLINE map2 #-}
map3 :: (a -> b) -> List' a -> List' b
map3 f = fromCoCh . natCoCh (m f) . toCoCh
{-# INLINE map3 #-}
\end{code}
\paragraph{Sum}
We define our sum function in, \textit{again} three different ways:
unencoded, Church encoded, and Cochurch encoded.
The non-encoded leverages simple recursion:
\begin{code}
sum1 :: List' Int -> Int
sum1 Nil = 0
sum1 (Cons x xs) = x + sum1 xs
{-# INLINE sum1 #-}
\end{code}
The Church encoded function leverages an algebra and applies that to the existing recursion principle:
\begin{code}
sum2 :: List' Int -> Int
sum2 = sumCh . toCh
  where sumCh :: ListCh Int -> Int
        sumCh (ListCh g) = g su
        su :: List_ Int Int -> Int
        su Nil_ = 0
        su (Cons_ x y) = x + y
{-# INLINE sum2 #-}

sum7 :: List' Int -> Int
sum7 = flip sumCh 0 . toCh
  where sumCh :: ListCh Int -> (Int -> Int)
        sumCh (ListCh g) = g su
        su :: List_ Int (Int -> Int) -> (Int -> Int)
        su Nil_ s = s
        su (Cons_ x y) s = y (s + x)
{-# INLINE sum7 #-}
\end{code}
A second recursion principle is also implemented that modifies the type of the recursion element in the base functor.
Leveraging call arity techniques as described and made possible by \cite{Breitner2018} to obtain a tail recursive implementation of sum for Church encodings.

The Cochurch encoded function implements a corecursion principle and applies the existing coalgebra (and input) to it:
\begin{code}
sum3 :: List' Int -> Int
sum3 = sumCoCh . toCoCh
  where sumCoCh :: ListCoCh Int -> Int
        sumCoCh (ListCoCh h s) = su h s
        su :: (s -> List_ Int s) -> s -> Int
        su h s = loopt s 0
          where loopt s' sum = case h s' of
                  Nil_ -> sum
                  Cons_ x xs -> loopt xs (x + sum)
{-# INLINE sum3 #-}

sum8 :: List' Int -> Int
sum8 = sumCoCh . toCoCh
  where sumCoCh :: ListCoCh Int -> Int
        sumCoCh (ListCoCh h s) = su h s
        su :: (s -> List_ Int s) -> s -> Int
        su h s = loop s
          where loop s' = case h s' of
                  Nil_ -> 0
                  Cons_ x xs -> x + loop xs
{-# INLINE sum8 #-}
\end{code}
Note that two subfunctions are provided to \tt{su'}, the \tt{loop} and the \tt{loopt} function.
The former function is implemented using typical recursion.
The latter, interestingly, is implemented using tail-recursion.
Because this \tt{loopt} function constitutes a corecursion principle, all the algebras (or natural transformations) applied to it, will be inlined in such a way that the resultant function is also tail recursive, in some cases providing a significant speedup!
For more details, see the discussion in \autoref{sec:tail}.

\paragraph{Pipelines and GHC list fusion}
Now we can make a pipeline in the following fashion:
\begin{code}
pipeline1 :: (Int, Int) -> Int
pipeline1 = sum1 . map1 (+2) . filter1 trodd . between1
\end{code}
You may notice I have not yet discussed the filter function, this is for a good reason, which I will discuss now.
\ignore{
\begin{code}
trodd :: Int -> Bool
trodd n = n `rem` 2 == 0
{-# INLINE trodd #-}

pipeline2 = sum2 . map2 (+2) . filter2 trodd . between2
pipeline7 = sum7 . map2 (+2) . filter2 trodd . between2
pipeline3 = sum3 . map3 (+2) . filter3 trodd . between3
pipeline8 = sum8 . map3 (+2) . filter3 trodd . between3
pipeline4 (x, y) = loop x 0
  where loop z sum = if z > y 
                     then sum
                     else if trodd z
                          then loop (z+1) (sum+z+2)
                          else loop (z+1) sum

between5 :: (Int, Int) -> [Int]
between5 (x, y) = [x..y]
{-# INLINE between5 #-}
filter5 :: (Int -> Bool) -> [Int] -> [Int]
filter5 f xs = build (\c n -> foldr (\a b -> if f a then c a b else b) n xs)
{-# INLINE filter5 #-}
map5 :: forall a b . (a -> b) -> [a] -> [b]
map5 f xs = build (\c n -> foldr (\a b -> c (f a) b) n xs)
{-# INLINE map5 #-}
sum5 :: [Int] -> Int
sum5 = foldl' (\a b -> a+b) 0
{-# INLINE sum5 #-}
pipeline5 = sum5 . map5 (+2) . filter5 trodd . between5
pipeline6 = sum6 . map6 (+2) . filter6 trodd . between6
pipeline9 = sum9 . map6 (+2) . filter6 trodd . between6
pipeline10 = sum10 . map10 (+2) . filter10 trodd . between10
pipeline11 = sum11 . map10 (+2) . filter10 trodd . between10
\end{code}
}

\paragraph{Filter}
The filter function is, again, implemented in three different ways:
In a non-encoded fashion, using a Church encoding, and using a Cochurch encoding.
The non-encoded function simply uses recursion:
\begin{code}
filter1 :: (a -> Bool) -> List' a -> List' a
filter1 _ Nil = Nil
filter1 p (Cons x xs) = if p x then Cons x (filter1 p xs) else filter1 p xs
{-# INLINE filter1 #-}
\end{code}
However, the (Co)Church encoded version, contrary to \tt{map}, can not be implemented using a natural transformation.

The following section will start to answer the following research question:
What tools and techniques are available to get Haskell's compiler to cooperate and trigger fusion?

\subsubsection{The Filter Problem}\label{sec:filter_prob}
There are multiple ways of implementing Church and Cochurch encoded filter functions, none of them immediately obvious from \cite{Harper2011}'s description of how it should be implemented as a natural transformation.

When replicating Harper's code for lists, there is one major limitation on natural transformation functions:
How to represent filter as a natural transformation for both Church and Cochurch encodings?
In his work he implemented, using Leaf trees, a natural transformation for the filter function in the following manner:
\begin{spec}
filt :: (a -> Bool) -> Tree_ a c -> Tree_ a c
filt p Empty_ = Empty_
filt p (Leaf_ x) = if p x then Leaf_ x else Empty_
filt p (Fork_ l r) = Fork_ l r
filter2 :: (a -> Bool) -> Tree a -> Tree a
filter2 p = fromCh . natCh (filt p) . toCh
filter3 :: (a -> Bool) -> Tree a -> Tree a
filter3 p = fromCoCh . natCoCh (filt p) . toCoCh
\end{spec}
This \tt{filt} function was then subsequently used in the Church and Cochurch encoded function.
Let us try this for the \tt{List} datatype:
\begin{spec}
filt :: (a -> Bool) -> List_ a c -> List_ a c
filt p Nil_ = Nil_
filt p (Cons_ x xs) = if p x then Cons_ x xs else ? 
\end{spec}
The question is, what should be in the place of the \tt{?} above?
Initially one might say \tt{xs}, as the \tt{Cons\_ x} part should be filtered away, and this would be conceptually correct except for the fact that \tt{xs} is of type \tt{c}, and not of type \tt{List\_ a c}.
Filling in \tt{xs} gives a type error.
We could try to modify the type to allow this change, but if we did that we wouldn't have the type of a natural transformation anymore, so we can't do that either.

There are two solutions:
One that modifies the definition of \tt{filter2} and \tt{filter3}, such that the definition is still possible, without leveraging natural transformations, instead creating a new algebra from an existing one.
The other modifies the definition of the underlying type such that the filter function is still possible to express as a transformation.
    
\paragraph{Solution 1: Abandoning Natural Transformations}
\subparagraph{Church} Before we wanted to implement our \tt{filter} function in the following manner:
\begin{spec}
filterCh :: (forall c . List_ a c -> List_ b c) -> ListCh a -> ListCh b
filterCh p (ListCh g) = ListCh (\a -> g (a . (filt p)))
\end{spec}
\begin{code}
filter2 :: (a -> Bool) -> List' a -> List' a
filter2 p = fromCh . filterCh p . toCh
{-# INLINE filter2 #-}
\end{code}
We now need to modify the \tt{filterCh} function such that we can still express a filter function \textit{without} using a natural transformation:
\begin{spec}
filterCh :: (forall c . List_ a c -> List_ b c) -> ListCh a -> ListCh b
filterCh p (ListCh g) = ListCh (\a -> g (?))
\end{spec}
Replacing the hole \tt{?} in the expression \tt{g (?)} above such that we apply the \tt{a} selectively we can yield:
\begin{code}
filterCh :: (a -> Bool) -> ListCh a -> ListCh a
filterCh p (ListCh g) = ListCh (\a -> g (\case
    Nil_ -> a Nil_
    Cons_ x xs -> if (p x) then a (Cons_ x xs) else xs
  ))
\end{code}
We create a new algebra from an existing one, \tt{a}, that selectively postcomposes \tt{a}, this is not a natural transformation anymore in the way that \tt{f} below is.
\begin{spec}
natCh :: (forall c . List_ a c -> List_ b c) -> ListCh a -> ListCh b
natCh f (ListCh g) = ListCh (\a -> g (a . f))
\end{spec}
In the new solution we do not apply \tt{a} to \tt{xs}, and, in doing so, can put \tt{xs} in the place where we wanted to earlier.
Before we were limited because the \tt{natCh} function forced a postcomposition of \tt{a} in all cases, which is now lifted by abandoning the \tt{natCh} function.

% I was just reading this: https://link.springer.com/chapter/10.1007/978-3-540-30477-7_22
% This is one of the fusion rules that is leveraged in GHC.List fusion.
% Interesting read.


\subparagraph{Cochurch}
Whereas before we wanted to implement our \tt{filter} function in the following manner:
\begin{spec}
filter3 :: (a -> Bool) -> List a -> List a
filter3 p = fromCoCh . natCoCh (filt p) . toCoCh
\end{spec}
For the Cochurch encoding, a natural transformation can be defined, but it is not a simple coalgebra, instead it is a recursive function.\footnote{And not necessarily guaranteed to terminate, the seed could generate an infinite structure.}
The core idea is: we combine the natural transformation and postcomposition again, but this time we make the function recursively grab elements from the seed until it finds one that satisfies the predicate.
\begin{code}
filt :: (a -> Bool) -> (s -> List_ a s) -> s -> List_ a s
filt p h s = go s
  where go s = case h s of
          Nil_ -> Nil_
          Cons_ x xs -> if p x then Cons_ x xs else go xs
filterCoCh :: (a -> Bool) -> ListCoCh a -> ListCoCh a
filterCoCh p (ListCoCh h s) = ListCoCh (filt p h) s
filter3 :: (a -> Bool) -> List' a -> List' a
filter3 p = fromCoCh . filterCoCh p . toCoCh
{-# INLINE filter3 #-}
\end{code}
The \tt{go} subfunction is recursive, so it does not inline (fuse) neatly into the main function body in the way that the rest of the pipeline does.
There is existing work, called join-point optimization that should enable this function to still fully fuse, but it does not at the moment.
There are existing issues in GHC's issue tracker that describe this problem.\footnote{\url{https://gitlab.haskell.org/ghc/ghc/-/issues/22227}}


\paragraph{Solution 2: go back and modify the underlying type}
It is possible to implement filter using a natural transformation, but this requires us to modify the type of the base functor.
We can add a new constructor to the datatype that allows us to null out the value of our datatype: \tt{ConsN'\_ xs}.
This way we can write the \tt{filt} function in the following fashion:
\begin{code}
data List'_ a b = Nil'_ | Cons'_ a b | ConsN'_ b
filt' :: (a -> Bool) -> List'_ a c -> List'_ a c
filt' p Nil'_ = Nil'_
filt' p (ConsN'_ xs) = ConsN'_ xs
filt' p (Cons'_ x xs) = if p x then Cons'_ x xs else ConsN'_ xs
\end{code}
Now we do need to modify all of our already defined functions to take into account this modified datatype.
Readers familiar with the work might notice that this technique is in fact \textit{stream fusion} as described by \cite{Coutts2007}.
The \tt{ConsN\_} constructor is analogous to the \tt{Skip} constructor.
Therefore, this is a known and understood technique, motivated by the limitations of the techniques described by Harper.

So why was it possible to implement \tt{filt} without modifying the datatype of leaf trees?
Because leaf trees already have this consideration of being able to null the datatype in-place by changing a \tt{Leaf\_ x} into an \tt{Empty\_}.
\tt{filt} is able to remove a value from the datastructure without changing the structure of the data i.e., it is still a natural transformation.
By changing the list datatype such that this nullability is also possible, we can now write \tt{filt} as a natural transformation.

This technique could be broader than a modification to just lists.
By modifying (making nullable) any datatype, it might be possible to broaden the class of functions that can be represented as a natural transformation.
One other example of this is already the difference between a \tt{Binary Tree} and a \tt{Leaf Tree} datatype:
\begin{spec}
data BinTree a = Leaf a | Fork (BinTree a) (BinTree a)
data LeafTree a = Empty | Leaf a | Fork (LeafTree a) (LeafTree a)
\end{spec}
The \tt{Leaf} constructor of \tt{BinTree} is made nullable.
We will leave the following question to future work: Is this generalizable to any datastructure (perhaps containers)?






\ignore{
\begin{code}
data ListCoCh' a = forall s . ListCoCh' (s -> List'_ a s) s
toCoCh' :: List' a -> ListCoCh' a
toCoCh' = ListCoCh' out
  where out :: List' a -> List'_ a (List' a)
        out Nil = Nil'_
        out (Cons x xs) = Cons'_ x xs
fromCoCh' :: ListCoCh' a -> List' a
fromCoCh' (ListCoCh' h s) = unfold h s
  where unfold :: (b -> List'_ a b) -> b -> List' a
        unfold h s = case h s of
          Nil'_ -> Nil
          ConsN'_ xs -> unfold h xs
          Cons'_ x xs -> Cons x (unfold h xs)
{-# RULES "toCh/fromCh fusion"
   forall x. toCoCh' (fromCoCh' x) = x #-}
{-# INLINE [0] toCoCh' #-}
{-# INLINE [0] fromCoCh' #-} 
betweenCoCh' :: (Int, Int) -> List'_ Int (Int, Int)
betweenCoCh' (x, y) = if (x > y)
                      then Nil'_
                      else Cons'_ x (x+1, y)
{-# INLINE betweenCoCh' #-}
between6 :: (Int, Int) -> List' Int
between6 = fromCoCh' . ListCoCh' betweenCoCh'
{-# INLINE between6 #-}
natCoCh' :: (forall c . List'_ a c -> List'_ b c) -> ListCoCh' a -> ListCoCh' b
natCoCh' f (ListCoCh' h s) = ListCoCh' (f . h) s
m' :: (a -> b) -> List'_ a c -> List'_ b c
m' f (Cons'_ x xs) = Cons'_ (f x) xs
m' _ (ConsN'_ xs) = ConsN'_ xs
m' _ Nil'_ = Nil'_
map6 :: (a -> b) -> List' a -> List' b
map6 f = fromCoCh' . natCoCh' (m' f) . toCoCh'
{-# INLINE map6 #-}
filter6 :: (a -> Bool) -> List' a -> List' a
filter6 p = fromCoCh' . natCoCh' (filt' p) . toCoCh'
{-# INLINE filter6 #-}
sum6 :: List' Int -> Int
sum6 = sumCoCh . toCoCh'
  where sumCoCh :: ListCoCh' Int -> Int
        sumCoCh (ListCoCh' h s) = su h s
        su :: (s -> List'_ Int s) -> s -> Int
        su h s = loopt 0 s
          where loopt sum s' = case h s' of
                  Nil'_ -> sum
                  ConsN'_ xs -> loopt sum xs
                  Cons'_ x xs -> loopt (x + sum) xs
{-# INLINE sum6 #-}
sum9 :: List' Int -> Int
sum9 = sumCoCh . toCoCh'
  where sumCoCh :: ListCoCh' Int -> Int
        sumCoCh (ListCoCh' h s) = su h s
        su :: (s -> List'_ Int s) -> s -> Int
        su h s = loop s
          where loop s' = case h s' of
                  Nil'_ -> 0
                  ConsN'_ xs -> loop xs
                  Cons'_ x xs -> x + loop xs
{-# INLINE sum9 #-}





data ListCh' a = ListCh' (forall b . (List'_ a b -> b) -> b)
toCh' :: List' a -> ListCh' a
toCh' t = ListCh' (\a -> fold' a t)
fold' :: (List'_ a b -> b) -> List' a -> b
fold' a Nil         = a Nil'_
fold' a (Cons x xs) = a (Cons'_ x (fold' a xs))
fromCh' :: ListCh' a -> List' a
fromCh' (ListCh' fold') = fold' in''
in'' :: List'_ a (List' a) -> List' a
in'' Nil'_ = Nil
in'' (ConsN'_ xs) = xs
in'' (Cons'_ x xs) = Cons x xs
{-# RULES "toCh/fromCh fusion"
   forall x. toCh' (fromCh' x) = x #-}
{-# INLINE [0] toCh' #-}
{-# INLINE [0] fromCh' #-}
natCh' :: (forall c . List'_ a c -> List'_ b c) -> ListCh' a -> ListCh' b
natCh' f (ListCh' g) = ListCh' (\a -> g (a . f))


b' :: (List'_ Int b -> b) -> (Int, Int) -> b
b' a (x, y) = loop x
  where loop x = case x > y of
          True -> a Nil'_
          False -> a (Cons'_ x (loop (x+1)))
betweenCh' :: (Int, Int) -> ListCh' Int
betweenCh' (x, y) = ListCh' (\a -> b' a (x, y))
{-# INLINE betweenCh' #-}
between10 :: (Int, Int) -> List' Int
between10 = fromCh' . betweenCh'
{-# INLINE between10 #-}
map10 :: (a -> b) -> List' a -> List' b
map10 f = fromCh' . natCh' (m' f) . toCh'
{-# INLINE map10 #-}
filter10 :: (a -> Bool) -> List' a -> List' a
filter10 p = fromCh' . natCh' (filt' p) . toCh'
{-# INLINE filter10 #-}
sum10 :: List' Int -> Int
sum10 = sumCh . toCh'
  where sumCh :: ListCh' Int -> Int
        sumCh (ListCh' g) = g su
        su :: List'_ Int Int -> Int
        su Nil'_ = 0
        su (ConsN'_ y) = y
        su (Cons'_ x y) = x + y
{-# INLINE sum10 #-}
sum11 :: List' Int -> Int
sum11 = flip sumCh 0 . toCh'
  where sumCh :: ListCh' Int -> (Int -> Int)
        sumCh (ListCh' g) = g su
        su :: List'_ Int (Int -> Int) -> (Int -> Int)
        su Nil'_ = oneShot (\s -> s)
        su (ConsN'_ y) = oneShot (\s -> y s)
        su (Cons'_ x y) = oneShot (\s -> y (s + x))
{-# INLINE sum11 #-}


makegroup n = [ 
      bench "pipunfused1" $ nf pipeline1 (1, n)
    , bench "pipunfused2" $ nf pipeline1 (1, n)
    , bench "pipunfused3" $ nf pipeline1 (1, n)
    , bench "pipunfused4" $ nf pipeline1 (1, n)
    , bench "pipunfused5" $ nf pipeline1 (1, n)
    , bench "pipchfused1" $ nf pipeline2 (1, n)
    , bench "pipchfused2" $ nf pipeline2 (1, n)
    , bench "pipchfused3" $ nf pipeline2 (1, n)
    , bench "pipchfused4" $ nf pipeline2 (1, n)
    , bench "pipchfused5" $ nf pipeline2 (1, n)
    , bench "pipchtailfused1" $ nf pipeline7 (1, n)
    , bench "pipchtailfused2" $ nf pipeline7 (1, n)
    , bench "pipchtailfused3" $ nf pipeline7 (1, n)
    , bench "pipchtailfused4" $ nf pipeline7 (1, n)
    , bench "pipchtailfused5" $ nf pipeline7 (1, n)
    , bench "pipchstreamfused1" $ nf pipeline10 (1, n)
    , bench "pipchstreamfused2" $ nf pipeline10 (1, n)
    , bench "pipchstreamfused3" $ nf pipeline10 (1, n)
    , bench "pipchstreamfused4" $ nf pipeline10 (1, n)
    , bench "pipchstreamfused5" $ nf pipeline10 (1, n)
    , bench "pipchstreamtailfused1" $ nf pipeline11 (1, n)
    , bench "pipchstreamtailfused2" $ nf pipeline11 (1, n)
    , bench "pipchstreamtailfused3" $ nf pipeline11 (1, n)
    , bench "pipchstreamtailfused4" $ nf pipeline11 (1, n)
    , bench "pipchstreamtailfused5" $ nf pipeline11 (1, n)
    , bench "pipcofused1" $ nf pipeline8 (1, n)
    , bench "pipcofused2" $ nf pipeline8 (1, n)
    , bench "pipcofused3" $ nf pipeline8 (1, n)
    , bench "pipcofused4" $ nf pipeline8 (1, n)
    , bench "pipcofused5" $ nf pipeline8 (1, n)
    , bench "pipcotailfused1" $ nf pipeline3 (1, n)
    , bench "pipcotailfused2" $ nf pipeline3 (1, n)
    , bench "pipcotailfused3" $ nf pipeline3 (1, n)
    , bench "pipcotailfused4" $ nf pipeline3 (1, n)
    , bench "pipcotailfused5" $ nf pipeline3 (1, n)
    , bench "pipcostreamfused1" $ nf pipeline9 (1, n)
    , bench "pipcostreamfused2" $ nf pipeline9 (1, n)
    , bench "pipcostreamfused3" $ nf pipeline9 (1, n)
    , bench "pipcostreamfused4" $ nf pipeline9 (1, n)
    , bench "pipcostreamfused5" $ nf pipeline9 (1, n)
    , bench "pipcostreamtailfused1" $ nf pipeline6 (1, n)
    , bench "pipcostreamtailfused2" $ nf pipeline6 (1, n)
    , bench "pipcostreamtailfused3" $ nf pipeline6 (1, n)
    , bench "pipcostreamtailfused4" $ nf pipeline6 (1, n)
    , bench "pipcostreamtailfused5" $ nf pipeline6 (1, n)
    , bench "piplistfused1" $ nf pipeline5 (1, n)
    , bench "piplistfused2" $ nf pipeline5 (1, n)
    , bench "piplistfused3" $ nf pipeline5 (1, n)
    , bench "piplistfused4" $ nf pipeline5 (1, n)
    , bench "piplistfused5" $ nf pipeline5 (1, n)
    , bench "piphandfused1" $ nf pipeline4 (1, n)
    , bench "piphandfused2" $ nf pipeline4 (1, n)
    , bench "piphandfused3" $ nf pipeline4 (1, n)
    , bench "piphandfused4" $ nf pipeline4 (1, n)
    , bench "piphandfused5" $ nf pipeline4 (1, n)
    ]
main :: IO ()
main = defaultMain
  [
    bgroup "Filter pipeline 100" (makegroup 100),
    bgroup "Filter pipeline 1000" (makegroup 1000),
    bgroup "Filter pipeline 10000" (makegroup 10000),
    bgroup "Filter pipeline 100000" (makegroup 100000),
    bgroup "Filter pipeline 1000000" (makegroup 1000000),
    bgroup "Filter pipeline 10000000" (makegroup 10000000)
  ]
{- Initial report on core representation analysis for List datatypes,
 - In the core fused representation, lists are completely absent
   for both fused and cofused pipeline
 - The cofused function ends up pulling ahead slightly, not sure why.
   There could be two possible reasons for this:
   - When compiling with tastybench, the fused pipeline is not occupied
     with "unpacking" and "repacking" the first input integer. This difference
     is not present in the core representation without tastybench and just a simple print.
   - In the recursive call there is a lambda present in the body of the case, while in the cofused
     version it is just a DEFAULT declaration, returning the sum of some value.
 - For both there is further optimization possible removing some overhead:
   Passing around the 'y' of the tuple.  This is the case for both fused and cofused.
 - Furthremore, after implementing a 'hand-fused' version of the pipeline,
   it was discovered that it performed identically to the cofused implementation.
   - This 'hand fused' pipeline was then further optimized using tail recursion.
     This turned out to give another 40% speedup, a bummer when compared to cofusion
   - However, with a small change in the definition of the 'final' function of the cofused pipeline,
     su', such that it is also tail recursive gave an identical speedup to the cofused pipeline.
   - One interesting thing of note is that the Core representation of the cofused tai-recursive
     function does not make use of gotos like the hand-written one does. It does, however, seem
     to have identical performance.
   - An analogous change for the Church fused pipeline seems more difficult...
    -}
{- pipeline 3 Rec { 
$wloop :: Int# -> Int# -> Int#
$wloop
  = \ (ww :: Int#) (ww1 :: Int#) ->
      case ># ww ww1 of {
        __DEFAULT ->
          case remInt# ww 3# of wild {
            __DEFAULT ->
              case +# wild (andI# (-# 0# (andI# (<# ww 0#) 1#)) 3#) of {
                __DEFAULT -> $wloop (+# ww 1#) ww1;
                0# ->
                  case $wloop (+# ww 1#) ww1 of ww2 { __DEFAULT ->
                  +# 2# (+# ww2 ww)
                  }
              };
            0# ->
              case $wloop (+# ww 1#) ww1 of ww2 { __DEFAULT ->
              +# 2# (+# ww2 ww)
              }
          };
        1# -> 0#
      }
end Rec }
-}
{- pipeline2 Rec {
main_$s$wb :: Int# -> Int# -> Int
main_$s$wb
  = \ (sc :: Int#) (ww :: Int#) ->
      case ># sc ww of {
        __DEFAULT ->
          case remInt# sc 3# of wild {
            __DEFAULT ->
              case +# wild (andI# (-# 0# (andI# (<# sc 0#) 1#)) 3#) of {
                __DEFAULT -> main_$s$wb (+# sc 1#) ww;
                0# ->
                  case main_$s$wb (+# sc 1#) ww of { I# y -> I# (+# 2# (+# y sc)) }
              };
            0# ->
              case main_$s$wb (+# sc 1#) ww of { I# y -> I# (+# 2# (+# y sc)) }
          };
        1# -> lvl
      }
end Rec }
-}


{- Compile with tastybench -}
{- Cofused
Rec {
$wloop :: Int# -> Int# -> Int#
$wloop
  = \ (ww :: Int#) (ww1 :: Int#) ->
      case ># ww ww1 of {
        __DEFAULT ->
          case remInt# ww 3# of wild {
            __DEFAULT ->
              case +# wild (andI# (-# 0# (andI# (<# ww 0#) 1#)) 3#) of {
                __DEFAULT -> $wloop (+# ww 1#) ww1;
                0# ->
                  case $wloop (+# ww 1#) ww1 of ww2 { __DEFAULT ->
                  +# 2# (+# ww2 ww)
                  }
              };
            0# ->
              case $wloop (+# ww 1#) ww1 of ww2 { __DEFAULT ->
              +# 2# (+# ww2 ww)
              }
          };
        1# -> 0#
      }
end Rec }
 -}
 {- fused
Rec {
$s$wb :: Int -> Int# -> Int
$s$wb
  = \ (ww :: Int) (ww1 :: Int#) ->
      case ww of { I# x ->
      case ># x ww1 of {
        __DEFAULT ->
          case remInt# x 3# of wild1 {
            __DEFAULT ->
              case +# wild1 (andI# (-# 0# (andI# (<# x 0#) 1#)) 3#) of {
                __DEFAULT -> $s$wb (I# (+# x 1#)) ww1;
                0# ->
                  case $s$wb (I# (+# x 1#)) ww1 of { I# y -> I# (+# 2# (+# y x)) }
              };
            0# ->
              case $s$wb (I# (+# x 1#)) ww1 of { I# y -> I# (+# 2# (+# y x)) }
          };
        1# -> lvl
      }
      }
end Rec } 
 -}








{-
Chfused
Rec {
$s$wb :: Int -> Int# -> Int
$s$wb
  = \ (ww :: Int) (ww1 :: Int#) ->
      case ww of { I# x ->
      case ># x ww1 of {
        __DEFAULT ->
          case remInt# x 2# of {
            __DEFAULT -> $s$wb (I# (+# x 1#)) ww1;
            0# ->
              case $s$wb (I# (+# x 1#)) ww1 of { I# y -> I# (+# 2# (+# y x)) }
          };
        1# -> lvl
      }
      }
end Rec }

Rec {
$wloop :: Int# -> Int# -> Int#
$wloop
  = \ (ww :: Int#) (ww1 :: Int#) ->
      case ># ww ww1 of {
        __DEFAULT ->
          case remInt# ww 2# of {
            __DEFAULT -> $wloop (+# ww 1#) ww1;
            0# ->
              case $wloop (+# ww 1#) ww1 of ww2 { __DEFAULT ->
              +# 2# (+# ww2 ww)
              }
          };
        1# -> 0#
      }
end Rec }
-}
\end{code}
}
