\documentclass{article}
\usepackage[a4paper,margin=25mm,
 left=50mm,right=50mm
]{geometry}
\usepackage[utf8]{inputenc}
\usepackage{apacite}
\usepackage{natbib}
\usepackage{listings}
\usepackage{tikz-cd}
\usepackage{float}
\usepackage{amsmath}
%include polycode.fmt
\renewcommand{\tt}[1]{\texttt{#1}}


\usepackage{stmaryrd}
\newcommand{\catam}[1]{\llparenthesis #1 \rrparenthesis}
\newcommand{\anam}[1]{\llbracket #1 \rrbracket}


\title{Master's Thesis}
\author{Eben Rogers}

\begin{document}

\maketitle


%include lhs/sections/10_introduction.tex
%include lhs/sections/20_background.tex
\section{Formalization}\label{sec:formalization}
In \cite{Harper2011}'s work ``A Library Writer's Guide to Shortcut Fusion'', the practice of implementing Church and CoChurch encodings is described, as well a paper proof necessary to show that the encodings optimizations employed are correct.

In this section the work I have done to formalize these proofs in the programming language Agda is discussed, as well as additional proofs to support the claims made in the paper.

The code can be neatly presented in roughly 2 parts:
\begin{itemize}
  \item The proofs of the category theory truths described by \cite{Harper2011}.
  \item The proofs about the (Co)Church encodings, again as described by \cite{Harper2011}.
\end{itemize}

A note on imports: Imports are omitted in the agda code except when an import renames a construct it is importing, this is most prevalent for \tt{Category}, \tt{Data.W}, and \tt{Container}.


\subsection{Category Theory Formalization}
\subsubsection{funct}
This module contains some simple definition, utilized in both complimentary structures (cata-/anamorphisms, church/cochurch).
\input{sections/agda/funct/funext.tex}
\input{sections/agda/funct/endo.tex}

\subsubsection{init}
This module defines F-Algebras, a candidate initial object $\mu$, and catamorphisms, and proves initiality of $\mu$, the fusion properties, and the catamorphism laws.
\input{sections/agda/init/initalg.tex}
\input{sections/agda/init/fusion.tex}
\input{sections/agda/init/initial.tex}

\subsubsection{term}
This module defines F-CoAlgebras, a candidate terminal object $\nu$, and anamorphisms, and proves terminality of $\nu$, the fusion properties, and the anamorphism laws.
This module is the compliment of \tt{init}.
\input{sections/agda/term/termcoalg.tex}
\input{sections/agda/term/cofusion.tex}
\input{sections/agda/term/terminal.tex}
\subsection{Short cut fusion}
\subsubsection{Church encodings}
\input{sections/agda/church/defs.tex}
\input{sections/agda/church/proofs.tex}
\input{sections/agda/church/inst/list.tex}
\input{sections/agda/church/inst/free.tex}

\subsubsection{Cochurch encodings}
\input{sections/agda/cochurch/defs.tex}
\input{sections/agda/cochurch/proofs.tex}
\input{sections/agda/cochurch/inst/list.tex}

\iffalse
OLD
\subsection{Proofs from the category theory `truths'.}
My formalization of \cite{Harper2011}'s work is organized into three main parts:
\begin{itemize}
    \item[\textbf{funct}] EXPAND: Definition of Agda endofunctors (through the use of containers) and postulate of functional extensionality.
    \item[\textbf{init/term}] EXPAND: Definitions of initial/terminal (co)algebras, fusion, and some other category theory proofs
    \item[\textbf{(co)church}] EXPAND: Definitions of (Co)Church encodings and the formal proofs of \cite{Harper2011}'s work, including postulates for the free theorems used.
\end{itemize}
An extensive description of \textbf{init/term} will be discussed in section \ref{sec:cat_truths}.

\subsection{Proofs of the category theory truths.}\label{sec:cat_truths}
The proof of fusion and needed definitions is split into three parts and uses the agda-categories library: %TODO: CITE 
\begin{itemize}
    \item[\textbf{agda-categories}] EXPAND: Definition of Sets, initial/terminal, f-(co)algebra, f-(co)algebra categories
    \item[\textbf{endo}] EXPAND: Inside of funct, contains the definition of an endofunctor over Agda types, using containers
    \item[\textbf{initalg}] EXPAND: Definitions of $\alg{F}$, a candidate for an initial object in $\alg{F}$, and proof of initiality for said object.
    \item[\textbf{fusion}] EXPAND: Proof of the fusion property
\end{itemize}
For \textbf{initalg} and \textbf{fusion} there also exist complimentary proofs for terminal F-Coalgebras.

\paragraph{agda-categories}
\paragraph{Endo}
\paragraph{initalg/termcoalg}
\paragraph{(co)fusion}


\subsection{Example implementation and demonstration of the correctness of the fusion for that instance.}
\fi



\iffalse
Outline:
- Harper's work has some mathematics in it, I formalized it.
- The formalization was done in two parts:
  - The formalization of the mathematics itself
  - The formalization of the mathematics, embedded in categories to leverage the fusion property
- The Formalization also implemented an example datastructure to demonstrate the proof's applicability in practice.
\fi



\section{Haskell Optimizations}
In \cite{Harper2011}'s work there were still multiple open questions left regarding the exact mechanics of what Church and Cochurch encodings did while making their way through the compiler. Why are Cochurch encodings faster in some pipelines, but slower in others? etc.

In this section I'll describe my work replicating the fused Haskell code of the \cite{Harper2011}'s work and further optimization opportunities that were discovered along the way.

\subsection{Church encodings}
\subsection{Cochurch encodings}



\iffalse
One question that comes up is: Yes this fusion is nice, but how does the fused code actually provide a speedup, isn't the language already lazy and therefore not ripe for such a speedup? What are Haskell's other optimizations that come into play that pushes the shortcut fusion over the finish as a fast optimization?
\fi





\section{Conclusion and Future Work}



\subsection{Future Work}
\begin{itemize}
    \item Strengthen Agda's typechecker wrt implicit parameters
    \item Strengthen Agda's termination checker wrt corecursive datastructures
    \item Implement (co)church-fused versions of Haskell's library functions.
    \item Investigate if creating a language that has this fusion built-in natively can be compiled more efficiently
    \item Look into leveraging parametricity with agda, so no \tt{posulate}'s are needed.
\end{itemize}


\bibliographystyle{apacite}
\bibliography{references.bib}

\section{Outline}
\begin{itemize}
    \item Introduction
    \item Background
    \item Formalization work and structure
    \item Implementation of Haskell generator code?
    \item Conclusion
\end{itemize}

\section{Project plan}
\begin{itemize}
    \item \cite{Harper2011}'s guide for implementing shortcut fusion through Church encodings is useful.
    This paper aims to do the following:
    \begin{itemize}
        \item  Formalize the proofs present in \cite{Harper2011}'s work in Agda.
        \item  Investigate whether it is possible to mechanically generate Church encodings of arbitrary functors (initial algebra datastructures) in Haskell.
    \end{itemize}
\end{itemize}

\end{document}
