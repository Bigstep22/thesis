\section{Agda Formalization of the Optimization}\label{sec:formalization}
In \cite{Harper2011}'s work ``A Library Writer's Guide to Shortcut Fusion'', he describes the practice of implementing Church and Cochurch encodings, as well as a paper proof necessary to show that the encodings employed are correct.
I pose and answer the following question in this section:
Are the transformations used to enable fusion safe? Meaning:
\begin{itemize}[noitemsep]
  \item Do the transformations in Haskell preserve the semantics of the language?
  \item If the mathematics and the encodings are implemented in a dependently typed language, is it possible to prove them to be correct?
\end{itemize}

In this section we discuss the work that was done to formalize Harper's proofs in the programming language Agda, as well as additional proofs to support the claims made in the paper.
My code is represented in roughly 3 parts, once for Church and once for Cochurch encodings, each part builds on the previous one:
\begin{itemize}[noitemsep]
  \item The proofs of the category theory properties, such as initiality/terminality of datatypes and the reflection property.
  \item The definition and proofs about the (Co)Church encodings, again as described by Harper.
  \item An example implementation of the list datatype, using containers.
\end{itemize}

The Agda code makes use of two libraries: agda-stdlib\footnote{\url{https://github.com/agda/agda-stdlib}} v2.0 and agda-categories\footnote{\url{https://github.com/agda/agda-categories}} v0.2.0.

The discussion of my implementation can be found in \autoref{sec:agda_form_disc}

\subsection{Common definitions}
Both the Church (initial) and Cochurch (terminal) halves of the formalization use these definitions.
\input{sections/agda/funct/funext.tex}

\paragraph{Containers}
In the Agda formalization we need to represent functors.
While a \tt{RawFunctor} datatype does exist in Agda's stdlib, it does not provide the necessary data such that proofs can easily be done over it, such as the functor laws.

Instead, we opt to use Containers to represent strictly positive functors as described by \cite{Abbott2005}.
\input{sections/agda/examples/container.tex}

The main benefit of leveraging containers to represent functions is that it maintains positivity as well as that the functor laws are true by definition.
We will discuss the process for deriving a container from a given (polynomial) later on when we need to derive it for lists, in \autoref{label}.



\subsection{Church Encodings}
\subsubsection{Category Theory: Initiality}\label{sec:agda_init}
This section is about my formalization of \cite{Harper2011}'s work that proves the needed category theory that is to be used later on in the (Co)Church part of the formalization.
This section specifically defines the category of F-Algebras and proves initiality of \tt($\mu$F, in') (the universal properties of folds) and the fusion property.
\input{sections/agda/church/initial.tex}


\subsubsection{Fusion: Church encodings}
This section focuses on the fusion of Church encodings, leveraging parametricity (free theorems) \citep{Wadler1989}.
\input{sections/agda/church/defs.tex}
\input{sections/agda/church/proofs.tex}
\subsubsection{Example: Church Encoded List fusion}\label{sec:agda_church_list}
\input{sections/44_deriving_container.tex}
\input{sections/agda/church/inst/list.tex}
% \input{sections/agda/church/inst/free.tex}


\subsection{Cochurch Encodings}
\subsubsection{Category Theory: Terminality}
This section specifically defines the category of F-Coalgebras and proves terminality of \tt{$\nu$F, out} (the universal properties of unfolds) and the fusion property.
This section is the complement of \autoref{sec:agda_init}.
\input{sections/agda/cochurch/terminal.tex}


\subsubsection{Fusion: Cochurch encodings}
This section focuses on the fusion of Cochurch encodings, leveraging parametricity (free theorems) and the fusion property.
\input{sections/agda/cochurch/defs.tex}
\input{sections/agda/cochurch/proofs.tex}
\subsubsection{Example: Cochurch Encoded List fusion}
\input{sections/agda/cochurch/inst/list.tex}


\subsection{Discussion of Agda Formalization}\label{sec:agda_form_disc}
I formalized that, given parametricity, the fusion of (Co)Church encodings are the same as their non-encoded counterpart as proved by \cite{Harper2011}.

This was done through multiple steps that build on each other, each step leaning on proofs and definitions from the previos one:
I did this by first proving initality of \tt{W} ($\mu$) types and terminality of \tt{M} ($\nu$) types.
Then I formalized the categorical fusion property, which only ended up being used in the proofs for fusion of Cochurch encodings.
Then I defined the Church and Cochurch encodings, along with their associated conversion functions.
After defining all of this, I formalized Harper's proof that shortcut fusion is possible for both Church and Cochurch encodings.

Building on this, I implemented the \tt{List} datastructure using containers.
Across this datastructure I implemented normal and (Co)Church encoded functions across these lists: \tt{between}, \tt{map}, \tt{filter}, and \tt{sum}.

Repeating my qustion:
Are the transformations used to enable fusion safe? Meaning:
\begin{itemize}[noitemsep]
  \item Do the transformations in Haskell preserve the semantics of the language?
  \item If the mathematics and the encodings are implemented in a dependently typed language, is it possible to prove them to be correct?
\end{itemize}
The first question can be answered as a conditional: Yes, as long as Haskell's type system contains parametricity.
The second question can be answered as: Yes it is possible, with some limitations, which are discussed below.
The question as a whole can be answered as a tentative `yes', keeping in mind the weaknesses discussed below.


\paragraph{Remaining Weaknesses}
There are two main remaining weaknesses in my current work:
First, the proof of terminality of terminal coalgebras is currently not terminating.
Second, the free theorems are currently postulated to be true instead of being proven to be true.

\subparagraph{Termination Checking}
I made an attempt to construct a terminating proof of terminality of \tt{M} types in agda through the use of a bisimulation, but due to time constraints I reverted to a version of the code that type checked, but did not terminate for a few proofs.
The functions are currently proved using a postulate called \tt{out-injective} that postulates that the constructor of the coinductive datatype is injective.
The above three functions are now non-terminating in the final state of the code.
Furthermore, the implementation of the Cochurch encoded list \tt{sum} function also was set to be non-terminating.
The state of the code before reverting can be seen in \autoref{app:bisim}.

\subparagraph{Postulates}
There are currently four postulates in the codebase.
I'll go through them in increasing order of noteworthiness:
\begin{itemize}[noitemsep]
  \item Functional extensionality.
  I used functional extensionality extensively throughout the repository.
  Its use is well-understood to be consistent and is provable from within cubical Agda.
  \item \tt{out}-injectivity.  Injectivity of coinductive datatypes is not supported out-of-the-box in Agda for good reason.
  However, it is needed for my type checking of the proofs of terminality, without the use of a bisimilarity.
  It exists to patch over the larger problem of termination checking above.
  If a bisimilarity relation were to be introduced, it can be removed.
  \item Two free theorems.
  The postulating of the free theorems was needed as it is currently not possible to prove the correctness of free theorems from within Agda.
  New research does exist by \cite{Muylder2024} that would enable the proving of the two free theorems, using internalized parametricity as originally described by \cite{Bernardy2012}.
  Doing so falls outside the scope of this research and is left to future work.
\end{itemize}

My Agda formalization has shown that Harper's work is correct, with some limitation, namely with respect to the proof of terminality of $M$.
It can be clearly seen, through the proof of some of the lemmas, that the fusion does not destroy the semantics of the functions being fused; that the fusion is correct.
There are multiple of future avenues that this research can take, this is discussed more extensively in \autoref{sec:conclusion}.

