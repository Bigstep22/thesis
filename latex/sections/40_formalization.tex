\section{Agda Formalization of the Optimization}\label{sec:formalization}
In \cite{Harper2011}'s work ``A Library Writer's Guide to Shortcut Fusion'', he describes the practice of implementing Church and Cochurch encodings, as well a paper proof necessary to show that the encodings optimizations employed are correct.

In this section I discuss the work I have done to formalize these proofs in the programming language Agda, as well as additional proofs to support the claims made in the paper.

The code can be presented in roughly 2 parts:
\begin{itemize}
  \item The proofs of the category theory truths, as described by Harper
  \item The proofs about the (Co)Church encodings, again as described by Harper
\end{itemize}

The Agda code makes use of two libraries:
\begin{itemize}[noitemsep]
  \item agda-stdlib\footnote{\url{https://github.com/agda/agda-stdlib}} v2.0
  \item agda-categories\footnote{\url{https://github.com/agda/agda-categories}} v0.2.0
\end{itemize}

\subsection{Common definitions}
Both the Church and Cochurch side use these definitions:
\input{sections/agda/funct/funext.tex}
This concludes the common definitions part of the Agda code, both the Church (initial) and Cochurch (terminal) halves of the formalization will use these definitions.

\subsection{Category Theory: Initiality}\label{sec:agda_init}
This section is about my formalization of \cite{Harper2011}'s work that proves the needed category theory that is to be used later on in the (Co)Church part of the formalization.
This section specifically defines the category of F-Algebras and proves initiality of \tt($\mu$F, in') (the universal properties of folds) and the fusion property.
\input{sections/agda/church/initial.tex}

\subsection{Category Theory: Terminality}
This section specifically defines the category of F-Coalgebras and proves terminality of \tt{$\nu$F, out} (the universal properties of unfolds) and the fusion property.
This section is the compliment of \autoref{sec:agda_init}.
\input{sections/agda/cochurch/terminal.tex}

\subsection{Fusion: Church encodings}
This section focuses on the fusion of Church encodings, leveraging parametricity (free theorems).
\input{sections/agda/church/defs.tex}
\input{sections/agda/church/proofs.tex}
\input{sections/agda/church/inst/list.tex}
\input{sections/agda/church/inst/free.tex}


\subsection{Fusion: Cochurch encodings}
This section focuses on the fusion of Cochurch encodings, leveraging parametricity (free theorems) and the fusion property.
\input{sections/agda/cochurch/defs.tex}
\input{sections/agda/cochurch/proofs.tex}
\input{sections/agda/cochurch/inst/list.tex}

\subsection{Discussion of Agda Formalization}
I formalized that, given parametricity, the fusion of (Co)Church encodings are the same as their non-encoded counterpart as proved by \cite{Harper2011}.

I did this by first proving initality of \tt{W} types (and terminality of \tt{M} types).
Then I formalized the categorical fusion property, which only ended up being used in the proofs for fusion of Cochurch encodings.
Then I defined the Church and Cochurch encodings, along with their associated conversion functions.
After defining all of this, I formalized Harper's proof that shortcut fusion is possible for both Church and Cochurch encodings.

Building on this, I implemented a \tt{List} datastructure using containers.
Across this datastructure I implemented normal and (Co)Church encoded functions across these lists: \tt{between}, \tt{map}, and \tt{sum}.

\paragraph{Remaining Weaknesses}
There are two main remaining weaknesses in my current work:
First, the proof of terminality of terminal coalgebras is currently not terminating.
Second, the free theorems are currently postulated to be true instead of being proven to be true.

\subparagraph{Termination Checking}
I made a serious attempt to prove the terminality of \tt{M} types in agda through the use of a bisimilation relation, but at the cutoff moment for the work there was still too much work remaining to warrant continuing it:
\begin{itemize}
  \item The reflection law was proven (as a bisimilarity)
  \item The computation law was proven (as a propositional equality)
  \item The termination of the `proof of uniqueness' part of the proof of terminality (also as a bisimilarity)
  \item The plan and execution of restructuring the further code that rests on the above proofs.
  Propositional equalities need to instead use some combination of the bisimilaity and propositional equality in Agda.
\end{itemize}
Instead the functions are now proved using a postulate called \tt{out-injective} that postulates that the coinductive datatype is injective.
The above three functions are now non-terminating in the final state of the code.
Furthermore, the implementation of the Cochurch encoded list \tt{sum} function also was set to be non-terminating.

\subparagraph{Postulates}
There are currently four postulates in the codebase.
I'll go through them in increasing order of noteworthiness:
\begin{itemize}
  \item Functional extensionality.
  I used functional extensionality extensively throughout the repository.
  Its use is well-understood and is provable within cubical Agda.
  \item \tt{out}-injectivity.
  Injectivity of coinductive datatypes is likely not supported out-of-the-box in Agda for good reason.
  It is needed for the type checking of the proofs of terminality.
  It exists to patch over the larger problem of termination checking above.
  If a bisimilarity relation were to be introduced, it can be removed.
  \item Two free theorems.
  The postulating of the free theorems was needed as it is currently not possible to prove the correctness of free theorems from within Agda.
  New research does exist by \cite{Muylder2024} that would enable the proving of the two free theorems.
  Doing so falls outside the scope of this research and is left to future work.
\end{itemize}

\paragraph{Future work}

