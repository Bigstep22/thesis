
%%%%% Dive into description of library writer's guide to shortcut fusion
\subsection{Library Writer's Guide to Shortcut Fusion}
Now that the sufficient category theory has been explained, it is possible to describe \cite{Harper2011}'s paper, which my thesis centers on called ``A Library Writer's Guide to Shortcut Fusion''.

In the work, \cite{Harper2011} explain the concept of Church and Cochurch encodings in four steps:
The necessary underlying category theory, the concepts of encodings and the proof obligations necessary for ensuring correctness of the encodings, the concepts of (Co)Church encodings with the proof of correctness, and finally an example implementation for leaf trees.
I will now go through each step briefly.

\subsubsection{Category Theory}
For the full overview of the category theory, see \autoref{sec:catetgory_theory}.
The main concepts that \cite{Harper2011} explains are the \textit{universal property of (un)folds}, the \textit{fusion law}, and the \textit{reflection law}; all of which can be derived from the category theory already described earlier.
\begin{alignat*}{3}
\text{The universal property of folds is as follows:} \\
h &= \catam{a} \iff && h \circ in &&= a \circ F h \\
\text{The fusion law as:} \\
h \circ \catam{a} &= \catam{b}~\Longleftarrow && h \circ a &&= b \circ F h \\
\text{And the reflection law as:} \\
\catam{in} &= id
\end{alignat*}
I formalized and proved all of these properties in my Agda formalization.
It is also interesting to note that, for the universal property of unfolds, the forward direction is the proof of existence and the backward direction the proof of uniqueness, for the proof of initiality of an algebra.
Converse definitions exist for terminal coalgebras, but I will not cover them in this section.
They do exist in my formalization.

\subsubsection{Encodings}
The purpose of the encodings is to encode recursive functions, which are not inlined by Haskell's optimizer, into ones that are capable of being inlined and therefore fused:
``For example, assume that we want to convert values of the recursive datatype \tt{$\mu$F} to values of a type \tt{F}.  The idea is that \tt{C} can faithfully represent values of \tt{$\mu$F}, but composed functions over \tt{C} can be fused automatically'' \citep{Harper2011}.

Now, instead of writing functions over \tt{$\mu$F}, we write functions over \tt{C}, along with two conversion functions \tt{con: $\mu$F $\to$ C} (concrete) and \tt{abs : C $\to$ $\mu$F} (abstract).
In order for the datatype \tt{C} to faithfully represent \tt{$\mu$F}, we need $abs \circ con = id_{\mu F}$. I.e. that \tt{C} can represent all values of \tt{$\mu$F} uniquely.

In total there are four main proof obligations, the one mentioned above, as well as the commutation of the following three diagrams:

\begin{figure}[H]\hfill
\begin{tikzcd}
    \mu F  \arrow[d,"f",swap] & C \arrow[l,"abs"] \arrow[d, "f_C"] \\
    \mu F & C \arrow[l,"abs"]
\end{tikzcd}\hfill\null
\begin{tikzcd}
    S  \arrow[d,"p",swap] \arrow[dr,"p_C"] \\
    \mu F & C \arrow[l,"abs"]
\end{tikzcd}\hfill\null
\begin{tikzcd}
    \mu F  \arrow[d,"c",swap] \arrow[r,"con"] & C \arrow[dl, "c_C"] \\
    T
\end{tikzcd}\hfill\null
\end{figure}

In the second diagram, \tt{p} is a producer function, generating a recursive data structure from a seed of type \tt{S}.
In the third diagram, \tt{c} is a consumer function, consuming a recursive data structure to produce a value of type \tt{T}.

\subsubsection{(Co)Church Encodings}
Next, \cite{Harper2011} proposes two encodings, \tt{Church} and \tt{CoChurch}.

\paragraph{Church} \tt{Church} is defined (abstractly) as the following datatype:
\begin{code}
data Church F = Ch (forall A => (F A -> A) -> A)
\end{code}
\tt{Church} contains a recursion principle (often referred to as \tt{g} throughout this thesis).
With conversion and abstraction functions \tt{toCh} and \tt{fromCh}:
\begin{code}
toCh :: mu F -> Church F
toCh x = Ch (\ a -> fold a x)
fromCh :: Church F -> mu F
fromCh (Ch g) = g in
\end{code}
From these definitions, Harper proves the four proof obligations along with a fifth proof, proving the other composition of \tt{con} and \tt{abs} to be equal to \tt{id}, thereby showing isomorphism.
For the proof of transformers and \tt{con $\circ$ abs = id}, Harper makes use of the free theorem for the polymorphic recursion principle \tt{g}.
In all the five proofs for Church encodings, Harper does not use the fusion property.

\paragraph{Cochurch} \tt{CoChurch} is defined (abstractly) as the following datatype:
\begin{code}
data CoChurch' F = exists S => CoCh (S -> F S) S
\end{code}
An isomorphic definition which Harper later uses is the one I end up using in my formalization:
\begin{code}
data CoChurch F = forall S => CoCh (S -> F S) S
\end{code}
The Cochurch encoding encodes a coalgebra and a seed value together.
The conversion and abstraction functions, \tt{toCoCh} and \tt{fromCoCh}:
\begin{code}
toCoCh :: nu F -> CoChurch F
toCoCh x = CoCh out x
fromCoCh :: CoChurch F -> nu F
fromCoCh (CoCh h x) = unfold h x  
\end{code}
Similarly to his description of Church encodings, Harper proves the four proof obligations as well as the additional fifth one.
The \tt{con $\circ$ abs = id} proof, leverages the free theorem for the corecursion principle of the type \tt{CoChurch}.
The proof for natural transformations, however, does not use the free theorem and instead uses the fusion property for unfolds.

\subsubsection{Example implementation}
To tie it all together, Harper gives an example implementation of how one would implement the encodings described so far.
For this he uses Leaf Trees.
He implements four functions, \tt{between}, \tt{filter}, \tt{concat}, and \tt{sum}, as a normal, recursive function, in Church encoded form, and in Cochurch encoded form.

In doing so, he shows exactly how one goes from using the normal, recursive datatypes and function that are typically used in Haskell, to Church and Cochurch encoded versions.
To conclude the performance of different compositions of functions are compared to show the performance benefits and differences between the three different variants of functions.

\iffalse
\begin{itemize}
    \Item 
    % https://scholar.google.com/scholar?hl=en&as_sdt=2005&sciodt=0%2C5&cites=9372977837493231928&scipsc=1&q=church&btnG=
\end{itemize}
\fi
