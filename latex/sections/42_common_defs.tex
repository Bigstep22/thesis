\subsection{Common definitions}
Both the Church (initial) and Cochurch (terminal) halves of the formalization use these definitions.
\input{sections/agda/funct/funext.tex}

\paragraph{Containers}
In the Agda formalization we need to represent functors.
While a \tt{RawFunctor} datatype does exist in Agda's stdlib, it does not provide the necessary data such that proofs can easily be done over it, such as the functor laws.

Instead, we opt to use Containers to represent strictly positive functors as described by \cite{Abbott2005}.
\input{sections/agda/examples/container.tex}

The main benefit of leveraging containers to represent functions is that it maintains positivity as well as that the functor laws are true by definition.
We will discuss the process for deriving a container from a given (polynomial) later on when we need to derive it for lists, in \autoref{label}.


