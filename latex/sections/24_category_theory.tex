\subsection{The category theory}\label{sec:cat_theory}
In order to explain what an initial/terminal F-(co)algebra is, I'll first need to explain what a functor is and, more pressingly, what a category is.
The concept of catamorphisms and anamorphisms (folds and unfolds) will follow suit.
If you're familiar with category theory and these concepts, you can skip this section.
The mathematics described here are based on the lecture notes by \cite{Ahrens2022}.

\subsubsection{A Category}
A \textbf{category} $\mathcal{C}$ is a collection of four pieces of data satisfying three properties:
\newcommand{\homm}[2]{\tt{hom}_\mathcal{#1}(#2)}
\begin{enumerate}
    \item A collection of objects, denoted by $\mathcal{C}_0$
    \item For any given objects $X,Y \in \mathcal{C}_0$, a collection of morphisms from $X$ to $Y$, denoted by \tt{hom}$_\mathcal{C}(X,Y)$, which is called a \textit{hom-set}.
    \item For each object $X\in \mathcal{C}_0$, a morphism \tt{Id}$_X\in\homm{C}{X,X}$, called the \textit{identity morphism} on $X$.
    \item A binary operation: $(\circ)_{X,Y,Z}: \homm{C}{Y,Z}\to\homm{C}{X,Y}\to\homm{C}{X,Z}$, called the \textit{composition operator}, and written infix without the indices $X,Y,Z$ as in $g \circ f$.
\end{enumerate}
These pieces of data should satisfy the following three properties:
\begin{enumerate}
    \item (\textbf{Left unit law}) For any morphism $f\in\homm{C}{X,Y}$: \begin{align*} f\circ \tt{Id}_X = f \end{align*}
    \item (\textbf{Right unit law}) For any morphism $f\in\homm{C}{X,Y}$: \begin{align*} \tt{Id}_Y\circ f = f \end{align*}
    \item (\textbf{Associative law}) For any morphisms $f\in\homm{C}{X,Y},g\in\homm{C}{Y,Z}$, and $h\in\homm{C}{Z,W}$: \begin{align*} h\circ(g\circ f)=(h\circ g)\circ f \end{align*}
\end{enumerate}

\subsubsection{Initial/Terminal Objects}
Categories can contain objects that have certain (useful) properties.
Two of these properties are as follows:
\begin{itemize}
    \item[\textbf{initial}]  Let $\mathcal{C}$ be a category. An object $A\in\mathcal{C}_0$ is \textbf{initial} if there is exactly one morphism from $A$ to any object $B\in\mathcal{C}_0$: 
    \begin{align*}
        \forall A,B\in\mathcal{C}_0: \exists! \homm{C}{A,B} \Longrightarrow \textbf{initial}(A)
    \end{align*}
    \item[\textbf{terminal}]  Let $\mathcal{C}$ be a category. An object $A\in\mathcal{C}_0$ is \textbf{terminal} if there is exactly one morphism from any object $B\in\mathcal{C}_0$ to $A$:
    \begin{align*}
        \forall A,B\in\mathcal{C}_0: \exists! \homm{C}{B,A} \Longrightarrow \textbf{terminal}(A)
    \end{align*}
\end{itemize}
The proofs of initality and terminality require a proof that is split into two steps: A proof of existence (The $\exists$ part of $\exists!$) and a proof of uniqueness (The $!$ part of $\exists!$).
The former is usually done by construction, giving an example of a function that satisfies the property and the latter is usually done my assuming that another $\homm{C}{A,B}$ (for the initial case) exists and showing that it must be equal to the one constructed.

\subsubsection{Functors}
For a given category $\mathcal{C}, \mathcal{D}$, a \textbf{functor} from $\mathcal{C}$ to $\mathcal{D}$ consists of two pieces of data satisfying two properties:
\begin{enumerate}
    \item A function mapping objects in $\mathcal{C}$ to $\mathcal{D}$: \begin{align*} \mathcal{C}_0 \to \mathcal{D}_0 \end{align*}
    \item For each $X,Y\in\mathcal{C}_0$, a function mapping morphisms in $\mathcal{C}$ to morphisms in $\mathcal{D}$: \begin{align*} \homm{C}{X,Y} \to \homm{D}{F(X),F(Y)} \end{align*}
\end{enumerate}
These pieces of data should satisfy these two properties:
\begin{enumerate}
    \item (\textbf{Composition law}) for any two morphisms $f\in\homm{C}{X,Y},g\in\homm{C}{Y,Z}$: \begin{align*} F(g\circ f) = F g \circ F f \end{align*}
    \item (\textbf{Identity law}) For any $X\in\mathcal{C}_0$, we have: \begin{align*} F(\tt{Id}_X) = \tt{Id}_{F(X)} \end{align*}
\end{enumerate}
An \textbf{endofunctor} is a functor that maps objects back to the category itself i.e., $F:\mathcal{C}\to\mathcal{C}$.

\subsubsection{(Category of) F-(Co)Algebras}
Given an endofunctor $F:\mathcal{C}\to\mathcal{C}$, an \textbf{F-Algebra} consists of two pieces of data:
\begin{enumerate}
    \item An object $C\in \mathcal{C}_0$
    \item A morphism $\phi\in\homm{C}{F(C),C}$
\end{enumerate}

An \textbf{F-Algebra Homomorphism} is, given by two F-Algebras $(C,\phi),(D,\psi)$, and a morphism\\ $f\in\homm{C}{C,D}$, such that the following diagram commutes (i.e., $f\circ\phi=\psi\circ Ff$):
\begin{figure}[H]\vspace{-1em}\hfill
\begin{tikzcd}
    FC \arrow[r,"\phi"] \arrow[d,"F f",swap] & C \arrow[d, "f"] \\
    FD \arrow[r,"\psi"] & D
\end{tikzcd}\hfill\null
\end{figure}\vspace{-1em}

\newcommand{\alg}[1]{\mathcal{A}lg(#1)}
\newcommand{\coalg}[1]{\mathcal{C}o\mathcal{A}lg(#1)}

The \textbf{category of F-Algebras} denoted by $\alg{F}$ consists of (the needed) four pieces of data:
\begin{enumerate}
    \item The objects are F-Algebras
    \item The morphisms are F-Algebra homomorphisms
    \item The identity on $(C,\phi)$ is given by the identity $\tt{Id}_C$ in $\mathcal{C}$
    \item The composition is given by the composition of morphisms in $\mathcal{C}$
\end{enumerate}
These pieces of data should satisfy the usual category laws: left/right unit law and composition law.
Note how $\alg{F}$ makes use of the underlying category $\mathcal{C}$ of the functor to define its objects.
An $\alg{F}$ implicitly contains an underlying category in which its objects are embedded.

An \textbf{F-Coalgebra} consists of two pieces of data:
\begin{enumerate}
    \item An object $C\in \mathcal{C}_0$
    \item A morphism $\phi\in\homm{C}{C,F(C)}$
\end{enumerate}
F-Coalgebra homomorphisms and $\coalg{F}$ can be defined conversely as done for F-Algebras.

\subsubsection{Catamorphisms and Anamorphisms}
Given (if it exists) an initial F-Algebra $(\mu^F,in)$ in $\alg{F}$. We can know that (by definition), that for any other F-Algebra $(C,\phi)$, there exists a \textit{unique} morphism $\catam{\phi}\in\homm{C}{\mu^F,C}$ such that the following diagram commutes i.e., $\catam{\phi}\circ in = \phi\circ F\catam{\phi}$:
\begin{figure}[H]\vspace{-1em}\hfill
\begin{tikzcd}
    F\mu^F \arrow[r,"in"] \arrow[d,"F \catam{\phi}",swap] & \mu^F \arrow[d, "\catam{\phi}"] \\
    FC \arrow[r,"\phi"] & C
\end{tikzcd}\hfill\null
\end{figure}\vspace{-1em}
A morphism of the form $\catam{\phi}$ is called a \textbf{catamorphism}.

A converse definition of catamorphisms exists, for terminal objects in $\coalg{F}$ exists, called \textbf{anamorphisms}, denoted by $\anam{\phi}$

\subsubsection{Fusion property}\label{sec:fusion_prop}
Now for the definition I've been building to, \textbf{fusion}:
Given an endofunctor $F:\mathcal{C}\to\mathcal{C}$ and an initial algebra $(\mu^F,in)$ in $\alg{F}$. For any two F-Algebras $(C,\phi)$ and $(D,\psi)$ and morphism $f\in\homm{C}{C,D}$ we have a \textbf{fusion property}: \begin{align*} f\circ\phi=\psi\circ F(f) \Longrightarrow f\circ\catam{\phi}=\catam{\psi} \end{align*}
In English, if $f$ is an F-Algebra homomorphism, we can know that the composition of f and the catamorphism of $\phi$ equals the catamorphism of $\psi$ ($f\circ\catam{\phi}=\catam{\psi}$). We can fuse two functions into one! This is summarized in the following diagram:
\begin{figure}[H]\vspace{-1em}\hfill
\begin{tikzcd}
    F\mu^F \arrow[r,"in"]
           \arrow[d,"F \catam{\phi}",swap]
           \arrow[dd,"F\catam{\psi}",swap, bend right=70]
   & \mu^F \arrow[d, "\catam{\phi}"]
           \arrow[dd, "\catam{\psi}", bend left=70] \\
    FC \arrow[r,"\phi"]
       \arrow[d,"F f",swap]
   & C \arrow[d, "f"] \\
    FD \arrow[r,"\psi"] & D
\end{tikzcd}\hfill\null
\end{figure}\vspace{-2em}~\\
The top square commutes by initiality of \tt{($\mu$F, in)}. The bottom one is the precondition, and the right triangle is the fusion.

A converse definition of fusion can be made for terminal object in $\coalg{F}$.

Having described all of this category theory, you might have a natural question: How does this relate to foldr/build fusion?
To tie all this together, I will describe \cite{Harper2011}'s work in \autoref{sec:libfusion}, who discusses a more generalized form of foldr/build list fusion, allowing for a much broader class of datastructures through Church encodings. As well as a generalized form of destroy/unfoldr fusion through Cochurch encodings.

Before describing Harper's work, it is prudent to clearly show the correspondence between category theory terms and functional programming terms that I use interchangeably.
This can be seen in \autoref{tab:trans_table}
\begin{table}[h]
  \centering
  \begin{tabular}{|l|l|}\hline
    \textbf{category theory} & \textbf{functional programming} \\\hline
    catamorphism             & fold                            \\
    anamorphism              & unfold                          \\
    F-algebra                & algebra                         \\
    F-coalgebra              & coalgebra                       \\
    F-algebra initiality     & universal property of folds     \\
    F-coalgebra terminality  & universal property of unfolds   \\\hline
  \end{tabular}
  \caption{The above tables matches category theoretical terms to functional programming terms.}
  \label{tab:trans_table}
\end{table}
