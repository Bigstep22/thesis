\section{Formalization}
In \cite{Harper2011}'s work ``A Library Writer's Guide to Shortcut Fusion'', the practice of implementing Church and CoChurch encodings is described, as well a paper proof necessary to show that the encodings optimizations employed are correct.

In this section the work I have done to formalize these proofs in the programming language Agda is discussed, as well as additional proofs to support the claims made in the paper.

The code can be neatly presented in roughly 2 parts:
\begin{itemize}
  \item The proofs of the category theory truths described by \cite{Harper2011}.
  \item The proofs about the (Co)Church encodings, again as described by \cite{Harper2011}.
\end{itemize}

A note on imports: Imports are omitted in the agda code except when an import renames a construct it is importing, this is most prevalent for \tt{Category}, \tt{Data.W}, and \tt{Container}.


\subsection{Category Theory Formalization}
\subsubsection{funct}
This module contains some simple definition, utilized in both complimentary structures (cata-/anamorphisms, church/cochurch).
\input{sections/agda/funct/funext.tex}
\input{sections/agda/funct/endo.tex}

\subsubsection{init}
This module defines F-Algebras, a candidate initial object $\mu$, and catamorphisms, and proves initiality of $\mu$, the fusion properties, and the catamorphism laws.
\input{sections/agda/init/initalg.tex}
\input{sections/agda/init/fusion.tex}
\input{sections/agda/init/initial.tex}

\subsubsection{term}
This module defines F-CoAlgebras, a candidate terminal object $\nu$, and anamorphisms, and proves terminality of $\nu$, the fusion properties, and the anamorphism laws.
This module is the compliment of \tt{init}.
\input{sections/agda/term/termcoalg.tex}
\input{sections/agda/term/cofusion.tex}
\input{sections/agda/term/terminal.tex}
\subsection{Short cut fusion}
\subsubsection{Church encodings}
\input{sections/agda/church/defs.tex}
\input{sections/agda/church/proofs.tex}
\input{sections/agda/church/inst/list.tex}
\input{sections/agda/church/inst/free.tex}

\subsubsection{Cochurch encodings}
\input{sections/agda/cochurch/defs.tex}
\input{sections/agda/cochurch/proofs.tex}
\input{sections/agda/cochurch/inst/list.tex}

\iffalse
OLD
\subsection{Proofs from the category theory `truths'.}
My formalization of \cite{Harper2011}'s work is organized into three main parts:
\begin{itemize}
    \item[\textbf{funct}] EXPAND: Definition of Agda endofunctors (through the use of containers) and postulate of functional extensionality.
    \item[\textbf{init/term}] EXPAND: Definitions of initial/terminal (co)algebras, fusion, and some other category theory proofs
    \item[\textbf{(co)church}] EXPAND: Definitions of (Co)Church encodings and the formal proofs of \cite{Harper2011}'s work, including postulates for the free theorems used.
\end{itemize}
An extensive description of \textbf{init/term} will be discussed in section \ref{sec:cat_truths}.

\subsection{Proofs of the category theory truths.}\label{sec:cat_truths}
The proof of fusion and needed definitions is split into three parts and uses the agda-categories library: %TODO: CITE 
\begin{itemize}
    \item[\textbf{agda-categories}] EXPAND: Definition of Sets, initial/terminal, f-(co)algebra, f-(co)algebra categories
    \item[\textbf{endo}] EXPAND: Inside of funct, contains the definition of an endofunctor over Agda types, using containers
    \item[\textbf{initalg}] EXPAND: Definitions of $\alg{F}$, a candidate for an initial object in $\alg{F}$, and proof of initiality for said object.
    \item[\textbf{fusion}] EXPAND: Proof of the fusion property
\end{itemize}
For \textbf{initalg} and \textbf{fusion} there also exist complimentary proofs for terminal F-Coalgebras.

\paragraph{agda-categories}
\paragraph{Endo}
\paragraph{initalg/termcoalg}
\paragraph{(co)fusion}


\subsection{Example implementation and demonstration of the correctness of the fusion for that instance.}
\fi



\iffalse
Outline:
- Harper's work has some mathematics in it, I formalized it.
- The formalization was done in two parts:
  - The formalization of the mathematics itself
  - The formalization of the mathematics, embedded in categories to leverage the fusion property
- The Formalization also implemented an example datastructure to demonstrate the proof's applicability in practice.
\fi

