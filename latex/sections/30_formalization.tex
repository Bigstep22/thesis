\section{Formalization}
In \cite{Harper2011}'s work ``A Library Writer's Guide to Shortcut Fusion'', the practice of implementing Church and CoChurch encodings is described, as well a paper proof necessary to show that the encodings optimizations employed are correct.

In this section the work I have done to formalize these proofs in the programming language Agda is discussed, as well as additional proofs to support the claims made in the paper.

\subsection{Proofs from the category theory `truths'.}
My formalization of \cite{Harper2011}'s work is organized into three main parts:
\begin{itemize}
    \item[\textbf{funct}] EXPAND: Definition of Agda endofunctors (through the use of containers) and postulate of functional extensionality.
    \item[\textbf{init/term}] EXPAND: Definitions of initial/terminal (co)algebras, fusion, and some other category theory proofs
    \item[\textbf{(co)church}] EXPAND: Definitions of (Co)Church encodings and the formal proofs of \cite{Harper2011}'s work, including postulates for the free theorems used.
\end{itemize}
An extensive description of \textbf{init/term} will be discussed in section \ref{sec:cat_truths}.

\paragraph{funct}
\paragraph{init/term}
\paragraph{(co)church}

\subsubsection{funct}
funext, I'll skip endo for now, done in next section

\subsubsection{(co)church}
defs, proofs 1-5

\subsection{Proofs of the category theory truths.}\label{sec:cat_truths}
The proof of fusion and needed definitions is split into three parts and uses the agda-categories library: %TODO: CITE 
\begin{itemize}
    \item[\textbf{agda-categories}] EXPAND: Definition of Sets, initial/terminal, f-(co)algebra, f-(co)algebra categories
    \item[\textbf{endo}] EXPAND: Inside of funct, contains the definition of an endofunctor over Agda types, using containers
    \item[\textbf{initalg}] EXPAND: Definitions of $\alg{F}$, a candidate for an initial object in $\alg{F}$, and proof of initiality for said object.
    \item[\textbf{fusion}] EXPAND: Proof of the fusion property
\end{itemize}
For \textbf{initalg} and \textbf{fusion} there also exist complimentary proofs for terminal F-Coalgebras.

\paragraph{agda-categories}
\paragraph{Endo}
\paragraph{initalg/termcoalg}
\paragraph{(co)fusion}


\subsection{Example implementation and demonstration of the correctness of the fusion for that instance.}






\iffalse
Outline:
- Harper's work has some mathematics in it, I formalized it.
- The formalization was done in two parts:
  - The formalization of the mathematics itself
  - The formalization of the mathematics, embedded in categories to leverage the fusion property
- The Formalization also implemented an example datastructure to demonstrate the proof's applicability in practice.
\fi

