
%%%%% Dive into description of library writer's guide to shortcut fusion
\subsection{Library Writer's Guide to Shortcut Fusion}
Now that the sufficient category theory has been explained, it is possible to describe \cite{Harper2011}'s paper, which my thesis centers on called ``A Library Writer's Guide to Shortcut Fusion''.

In the work, \cite{Harper2011} explain the concept of Church and CoChurch encodings in three steps.
The necessary underlying category theory, the concepts of encodings and the proof obligations necessary for ensuring correctness of the encodings, and finally the concepts of (Co)Church encodings with the proof of correctness followed by an example implementation for leaf trees.
I will now go through each step briefly.

\subsubsection{Category Theory}
For the full overview of the category theory, see section \ref{sec:catetgory_theory}.
The main concepts that \cite{Harper2011} explains are the \textit{universal property of (un)folds}, the \textit{fusion law}, and the \textit{reflection law}; all of which can be derived from the category theory already described earlier.
\begin{alignat*}{3}
\text{The universal property of folds is as follows:} \\
h &= \catam{a} \iff && h \circ in &&= a \circ F h \\
\text{The fusion law as:} \\
h \circ \catam{a} &= \catam{b}~\Longleftarrow && h \circ a &&= b \circ F h \\
\text{And the reflection law as:} \\
\catam{in} &= id \\
\end{alignat*}



\iffalse
\begin{itemize}
    \item 
    % https://scholar.google.com/scholar?hl=en&as_sdt=2005&sciodt=0%2C5&cites=9372977837493231928&scipsc=1&q=church&btnG=
\end{itemize}
\fi
