
\subsection{Containers}
In my formalization I needed to represent functors somehow.
While a \tt{RawFunctor} datatype does exist, it does not provide the necessary structure such that proofs can be done over it, such as the functor laws.

Instead, I have opted to use Containers to represent strictly positive functors as described by \cite{Abbott2005}.
The definition of a container is as follows:
ADFSDFSDFKJSDHFSDKLJHF
% Description of container
These can be given a semantics (or extension) in the following manner:
ADFSDFSDFKJSDHFSDKLJHF
% Description of container 'interpretation'

The main benefit of leveraging containers to represent functions is that positivity is maintained as well as that the functor laws are true by definition.
Deriving the container from a given (polynomial) functor is done in a couple of steps:
\begin{enumerate}
    \item Analyze how many constructors your functor has, take as an example 2.
    \item For the left side of the container take the coproduct of types that store the non-recursive sub-elements (such as const).
    \item Count the amount of recursive elements in the constructor, the return type should include that many elements.
\end{enumerate}
Taking an example:
\begin{itemize}
    \item[\tt{List}]
    Taking the base functor for \tt{List}: \tt{F$_A$ X := 1 + A $\times$ X}.

    For the left side we take the coproduct of \tt{Fin 1} and \tt{const A}, corresponding to the `\tt{nil}' and `\tt{cons a \_}' part, respectively.

    For the right side, we have one constructor that is non-recursive and one that contains one recursive element, so we have:
    \tt{0 $\to$ Fin 0} and \tt{const n $\to$ Fin 1}.
    The Fin 1 refers to the recursive X that is present in the base functor (or the `\tt{cons \_ as}' part of cons).
    % Clean this up
    \item[\tt{Binary tree}]
    Taking the base functor for \tt{Tree}: \tt{F$_A$ X := 1 + X $\times$ A $\times$ X}.

    For the left side we take the coproduct of \tt{Fin 1} and \tt{const A}.

    For the right side, we have one constructor that is non-recursive and one that contains two recursive elements, so we have:
    \tt{0 $\to$ Fin 0} and \tt{const n $\to$ Fin 2}.
    % Clean this up
\end{itemize}
The above description is summarized below in a table:
\begin{table}[h]
  \begin{tabular}{|r|l|l|}\hline
     & \tt{List} &  \tt{Binary Tree} \\\hline
    Base functor         & \tt{F$_A$ X := 1 + (A $\times$ X)}  & \tt{F$_A$ X := 1 + (X $\times$ A $\times$ X)} \\\hline
    Left container half  & \tt{Fin 1 + const A} & \tt{Fin 1 + const A} \\\hline
    Right container half & \tt{nil $\to$ Fin 0 and const n $\to$ Fin 1} & \tt{nil $\to$ Fin 0 and const n $\to$ Fin 2} \\\hline
  \end{tabular}
\end{table}
\iffalse
- Strictly positive functors
\fi