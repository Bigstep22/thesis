\subsection{Performance Comparison}\label{sec:hask_perf}
I tested the performance of the following pipeline:
\begin{spec}
f = sum . map (+2) . filter odd . between
\end{spec}
I defined 11 different variants of the above functions which can be categorized into the following five groups:
\begin{itemize}[noitemsep]
\item Unfused
\item Hand fused
\item GHC List fused
\item Church fused (normally, with tail recursion\footnote{\href{https://gitlab.haskell.org/ghc/ghc/-/wikis/one-shot}{oneShot} needed to be used in order to get this to work.}, with skip constructor, and both)
\item Cochurch fused (normally, with tail recursion, with skip constructor, and both)
\end{itemize}

I'll discuss the results after first describing what is important when writings fusible functions.  For the implementation of all the functions, see the source code in the artifacts.

In order to make sure a pipeline of functions fuses in Haskell, there are several things that need to be taken into consideration.
\begin{itemize}[noitemsep]
    \item Make sure you only pass through parameters that change between recursive calls. Instead of writing:
    \begin{spec}
b a (x, y) = loop x y
  where loop x y = if x > y
                   then a Nil_
                   else a (Cons_ x (loop (x+1) y))
    \end{spec}
    Where the \tt{y} doesn't change between calls of \tt{loop}, modify \tt{loop} such that it doesn't pass through the \tt{y}:
    \begin{spec}
b a (x, y) = loop x
  where loop x = if x > y
                 then a Nil_
                 else a (Cons_ x (loop (x+1)))
    \end{spec}
    This way, fewer data need to be pushed around in memory for each (recursive) function call.
    \item Ensure that functions are inlined properly. So for the second example above add a pragma that inlines the function.
    This maximizes the time that Haskell can apply other optimizations between the bodies of multiple fusing functions:
    \begin{spec}
        {-# INLINE betweenCh #-}
    \end{spec}
    \item Ensure that the fused result is tail recursive.
    For consumer functions, it is often possible to make the function tail recursive.
    For the corecursion principle of sum \tt{su}:
    \begin{spec}
su :: (s -> List_ Int s) -> s -> Int
su h s = loop s
  where loop s' = case h s' of
          Nil_ -> 0
          Cons_ x xs -> x + loop xs
    \end{spec}
    It is possible to modify the definition of the corecursion \tt{loop} such that it is tail-recursive:
    \begin{spec}
su :: (s -> List_ Int s) -> s -> Int
su h s = loopt s 0
  where loopt s' sum = case h s' of
          Nil_ -> sum
          Cons_ x xs -> loopt xs (x + sum)
    \end{spec}
For Church encodings, it is a little more tricky to get the resultant function to be tail-recursive, it is possible, however.
Taking the algebra for sum again:
    \begin{spec}
sum2 :: List' Int -> Int
sum2 = sumCh . toCh
  where sumCh :: ListCh Int -> Int
        sumCh (ListCh g) = g su
        su :: List_ Int Int -> Int
        su Nil_ = 0
        su (Cons_ x y) = x + y
\end{spec}
We can modify the type of the recursive part of list and the return type to be a function instead of just a simple datatype (\tt{Int -> Int} instead of \tt{Int}):
\begin{spec}
sum7 :: List' Int -> Int
sum7 = flip sumCh 0 . toCh
  where sumCh :: ListCh Int -> (Int -> Int)
        sumCh (ListCh g) = g su
        su :: List_ Int (Int -> Int) -> (Int -> Int)
        su Nil_ s = s
        su (Cons_ x y) s = y (s + x)
    \end{spec}

This is technique similar to the one described by \cite{Breitner2018}.
\item Ensure that the fused result is a single recursive function (so no helper functions such as \tt{go}).
This was a problem when writing the filter function in Cochurch encoded fashion.
This is only possible if a recursive natural transformation function is used, but it unfortunately does not currently fuse.

The current workaround for this is Stream fusion as described by \cite{Coutts2007}.
This workaround is part of my performance analysis.

% Cue story about haskell's optimizer and that it is only willing to inline non-recursive functions. You only can recurse once... (Maybe something, something joing points?)
% Adding a Skip constructor can make a big difference to enabling the avoidance of additionally needed recursive functions.
\end{itemize}


For the testing I ensured that the first two of the above points were satisfied, partially through analysis of the GHC\footnote{\url{https://downloads.haskell.org/ghc/latest/docs/users_guide/debugging.html\#core-representation-and-simplification}} generated Core representation.
The latter two points became part of the testing setup.
I measured the performance using tasty-bench\footnote{\url{https://hackage.haskell.org/package/tasty-bench}}.
I tested all of the piplines with an input going from \tt{(1, 100)} to \tt{(1, 10000000)},
running tastybench five times for each input, setting a maximum standard deviation of 2\% of the mean result.
For the presentation of the data I took the mean of these five runs.
Tastybench keeps running tests until the standard deviation becomes small enough; each time running doubling the amount of runs before checking the new standard deviation.
Tastybench measured time using CPU time.

\subsubsection{Performance differences}
There are two main results figures, which can be seen at \autoref{fig:results1} and \autoref{fig:results2}.
However, their y axes are logarithmic, due to the nature of the input sizes provided from \tt{(1,100)} to \tt{(1,10000000)} in powers of 10.
It is more illustrative to look at a linear scale, and that is easiest when zooming in on one specific input.
For the illustration, I will choose \tt{(1,10000)} as input, as it is relatively representative.
There are specific variations when changing scale, but those will be discussed in \autoref{sec:scale_var}.

In \autoref{fig:res_1000} you can see how implementing fusion can bring quite a large speedup to a function pipeline.
With the following things of note:
\begin{itemize}[noitemsep]
    \item Tail recursive Church, stream Church, and stream Cochurch implementations were the fastest, and as fast as each other. A speedup of 25x over the unfused pipeline for this input.
    \item Stream fusion does not offer a speedup for Church encodings.
    \item Adding tail recursion speeds up the encoding in all cases, except:
    \item Church-encoded non-stream pipelines are not faster, this is due to a recursive natural transformation for filter (the function \tt{go}).
\end{itemize}

\begin{figure}[h]
    \includesvg[width=\textwidth]{figures/Execution time for input (1, 10000).svg}
    \caption{Comparison of execution times between the different pipelines.}
    \label{fig:res_1000}
\end{figure}

\subsubsection{Scale Variations}\label{sec:scale_var}
In general, as the scale increases (see \autoref{fig:all_exec} and \autoref{fig:all_exec_log} for all graphs):
\begin{itemize}[noitemsep]
    \item The factor speedup between the unfused and three fully fused pipelines increases, most notably between \tt{(1, 10000)} and \tt{(1,100000)}, going from 25.4x to 31.5x.
    This likely has to do with the increased volume of data that needs to be stored in random access memory.
    \item All the non tail-recursive encoding implementations get slower relative to the tail-recursive implementations.
    This is likely due to extra time spent allocating to and retrieving from memory.
    \item Most importantly for all fully fused pipelines: The speedup that fusion offers only seems to increase as the order of magnitude of the calculation grows.
\end{itemize}

\begin{figure}[H]
    \centering
    \includesvg[width=0.83\textwidth]{figures/Execution time for all pipelines.svg}
    \caption{Comparison of executions times between the different pipelines and input sizes, bar chart}
    \label{fig:results1}
\end{figure}
\begin{figure}[H]
    \centering
    \includesvg[width=0.83\textwidth]{figures/Execution time for all pipelines - stacked.svg}
    \caption{Comparison of executions times between the different pipelines and input sizes, line chart. This view makes it slightly easier to compare the differences between pipelines across different inputs.}
    \label{fig:results2}
\end{figure}
