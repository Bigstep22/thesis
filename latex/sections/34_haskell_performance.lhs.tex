\subsection{Performance Comparison}\label{sec:hask_perf}
We tested the performance of the following pipeline:
\begin{spec}
f = sum . map (+2) . filter odd . between
\end{spec}
We define 11 different variants of the above functions which can be categorized into the following five groups:
\begin{itemize}[noitemsep]
\item Unfused
\item Hand fused
\item GHC List fused
\item Church fused (normally, with tail recursion\footnote{\href{https://gitlab.haskell.org/ghc/ghc/-/wikis/one-shot}{oneShot} needed to be used in order to get this to work.}, with skip constructor, and both)
\item Cochurch fused (normally, with tail recursion, with skip constructor, and both)
\end{itemize}
For the implementation of all the functions, see the source code in the artifacts.

For the testing I ensured that the first two of the points in \ref{sec:hs_perf} were satisfied, partially through analysis of the GHC\footnote{\url{https://downloads.haskell.org/ghc/latest/docs/users_guide/debugging.html\#core-representation-and-simplification}} generated Core representation.
The latter two points became part of the testing setup.
I measured the performance using tasty-bench\footnote{\url{https://hackage.haskell.org/package/tasty-bench}}.
I tested all of the piplines with an input going from \tt{(1, 100)} to \tt{(1, 10000000)},
running tastybench five times for each input, setting a maximum standard deviation of 2\% of the mean result.
For the presentation of the data I took the mean of these five runs.
Tastybench keeps running tests until the standard deviation becomes small enough; each time running doubling the amount of runs before checking the new standard deviation.
Tastybench measured time using CPU time.

\subsubsection{Performance differences}
There are two main results figures, which can be seen at \autoref{fig:results1} and \autoref{fig:results2}.
However, their y axes are logarithmic, due to the nature of the input sizes provided from \tt{(1,100)} to \tt{(1,10000000)} in powers of 10.
It is more illustrative to look at a linear scale, and that is easiest when zooming in on one specific input.
For the illustration, I will choose \tt{(1,10000)} as input, as it is relatively representative.
There are specific variations when changing scale, but those will be discussed in \autoref{sec:scale_var}.

In \autoref{fig:res_1000} you can see how implementing fusion can bring quite a large speedup to a function pipeline.
With the following things of note:
\begin{itemize}[noitemsep]
    \item Tail recursive Church, stream Church, and stream Cochurch implementations were the fastest, and as fast as each other. A speedup of 25x over the unfused pipeline for this input.
    \item Stream fusion does not offer a speedup for Church encodings.
    \item Adding tail recursion speeds up the encoding in all cases, except:
    \item Church-encoded non-stream pipelines are not faster, this is due to a recursive natural transformation for filter (the function \tt{go}).
\end{itemize}

\begin{figure}[h]
    \includesvg[width=\textwidth]{figures/Execution time for input (1, 10000).svg}
    \caption{Comparison of execution times between the different pipelines.}
    \label{fig:res_1000}
\end{figure}

\subsubsection{Scale Variations}\label{sec:scale_var}
In general, as the scale increases (see \autoref{fig:all_exec} and \autoref{fig:all_exec_log} for all graphs):
\begin{itemize}[noitemsep]
    \item The factor speedup between the unfused and three fully fused pipelines increases, most notably between \tt{(1, 10000)} and \tt{(1,100000)}, going from 25.4x to 31.5x.
    This likely has to do with the increased volume of data that needs to be stored in random access memory.
    \item All the non tail-recursive encoding implementations get slower relative to the tail-recursive implementations.
    This is likely due to extra time spent allocating to and retrieving from memory.
    \item Most importantly for all fully fused pipelines: The speedup that fusion offers only seems to increase as the order of magnitude of the calculation grows.
\end{itemize}

These findings highlight how the fusion can provide a significant speedup to compiled Haskell code.
The replication of Harper's code show that achieving fused, tail-recursive compiled code requires taking into consideration many parts of Haskell's optimizer.
I believe that there exists big potential for Haskell's library writers to implement many datastructures using in a (Co)Church encodings; especially the foldr/build and destroy/unfoldr functions, from which many standard library functions and other functionalities can be derived.

\begin{figure}[H]
    \centering
    \includesvg[width=0.83\textwidth]{figures/Execution time for all pipelines.svg}
    \caption{Comparison of executions times between the different pipelines and input sizes, bar chart}
    \label{fig:results1}
\end{figure}
\begin{figure}[H]
    \centering
    \includesvg[width=0.83\textwidth]{figures/Execution time for all pipelines - stacked.svg}
    \caption{Comparison of executions times between the different pipelines and input sizes, line chart. This view makes it slightly easier to compare the differences between pipelines across different inputs.}
    \label{fig:results2}
\end{figure}

