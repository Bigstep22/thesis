\section{Related Works}\label{sec:related}
Initial work, done by \cite{Wadler1984,Wadler1986,Wadler1990} was dubbed `deforestation', referring to the removal of intermediate trees (or lists).
The details of the original deforestation work are not relevant to this thesis, but \cite{Gill1993} described the weaknesses of the work and proposed an alternative technique.
This so-called foldr/build fusion technique can, when employed, eliminate the runtime generation of intermediate lists.
I describe this technique further in \autoref{sec:foldr/build}.

A converse approach, aptly named the \tt{destroy/unfoldr} rule, is described by \cite{Svenningsson2002}, which describes the converse technique to \cite{Gill1993}'s.
A further generalization of this technique, dubbed \textit{stream fusion} by \cite{Coutts2007}, further strengthened the work by \cite{Svenningsson2002}.

\paragraph{(Co)Church encodings}
Finally, \cite{Harper2011} combined all of these concepts into one paper, called ``The Library Writer's Guide to Shortcut Fusion''.
In it the concept he describes (Co)Church encodings and, pragmatically, how to implement them in Haskell.

\paragraph{Other approaches}
Other approaches exist such as `Warm fusion' by \cite{Launchbury1995}, who attempt to derive fold and build combinators for a data type and automatically rewrite explicitly recursive functions.

Before \cite{Gill1993} published his work on shortcut fusion, there was existing work by \cite{Meijer1991}, describing the fusion properties of catamorphisms and anamorphisms, called ``Functional programming with bananas, lenses, envelopes and barbed wire''.