\section{Related Works}\label{sec:related}

\subsection{Fusion}
Initial work on fusion was done my \cite{Wadler1984,Wadler1986,Wadler1990}, and was dubbed `deforestation', referring to the removal of intermediate trees (or lists).
The details of the original deforestation work are not relevant to this thesis, but, the weaknesses of the work are described and different techniques proposed by \cite{Gill1993}.
\cite{Gill1993} describe a technique nowadays called \tt{foldr/build} fusion, which, when employed, can eliminate most intermediate lists.
This technique is described further in \autoref{sec:foldr/build}.

A converse approach, aptly named the \tt{destroy/unfoldr} rule, is described by \cite{Svenningsson2002}, which describes the converse technique to \cite{Gill1993}'s.
A further generalization of this technique, leverages the coinductive list datatype, stream. This technique is called \textit{stream fusion} introduced by \cite{Coutts2007}.

\paragraph{(Co)Church encodings}
Finally, \cite{Harper2011} combined all of these concepts into one paper, called ``The Library Writer's Guide to Shortcut Fusion''. In it the concept of (Co)Church encodings are described and, pragmatically, how to implement them in Haskell.