
\subsection{Discussion of Agda Formalization}\label{sec:agda_form_disc}
I formalized that, given parametricity, the fusion of (Co)Church encodings are the same as their non-encoded counterpart as proved by \cite{Harper2011}.

I did this by first proving initality of \tt{W} ($\mu$) types and terminality of \tt{M} ($\nu$) types.
Then I formalized the categorical fusion property, which only ended up being used in the proofs for fusion of Cochurch encodings.
Then I defined the Church and Cochurch encodings, along with their associated conversion functions.
After defining all of this, I formalized Harper's proof that shortcut fusion is possible for both Church and Cochurch encodings.

Building on this, I implemented a \tt{List} datastructure using containers.
Across this datastructure I implemented normal and (Co)Church encoded functions across these lists: \tt{between}, \tt{map}, and \tt{sum}.

Repeating my qustion:
Are the transformations used to enable fusion safe? Meaning:
\begin{itemize}[noitemsep]
  \item Do the transformations in Haskell preserve the semantics of the language?
  \item If the mathematics and the encodings are implemented in a dependently typed language, is it possible to prove them to be correct?
\end{itemize}
The first question can be answered as a conditional: Yes, as long as Haskell's type system contains parametricity.
The second question can be answered as: Yes it is possible, with some limitations, which are discussed below.
The question as a whole can be answered as a tentative `yes', again keeping in mind the weaknesses discussed below.


\paragraph{Remaining Weaknesses}
There are two main remaining weaknesses in my current work:
First, the proof of terminality of terminal coalgebras is currently not terminating.
Second, the free theorems are currently postulated to be true instead of being proven to be true.

\subparagraph{Termination Checking}
I made an attempt to construct a terminating proof of terminality of \tt{M} types in agda through the use of a bisimulation, but due to time constraints I reverted to a version of the code that type checked, but did not terminate for a few proofs.
The functions are currently proved using a postulate called \tt{out-injective} that postulates that the constructor of the coinductive datatype is injective.
The above three functions are now non-terminating in the final state of the code.
Furthermore, the implementation of the Cochurch encoded list \tt{sum} function also was set to be non-terminating.
The state of the code before reverting can be seen in \autoref{app:bisim}.

\subparagraph{Postulates}
There are currently four postulates in the codebase.
I'll go through them in increasing order of noteworthiness:
\begin{itemize}[noitemsep]
  \item Functional extensionality.
  I used functional extensionality extensively throughout the repository.
  Its use is well-understood to be consistent and is provable from within cubical Agda.
  \item \tt{out}-injectivity.  Injectivity of coinductive datatypes is not supported out-of-the-box in Agda for good reason.
  However, it is needed for my type checking of the proofs of terminality, without the use of a bisimilarity.
  It exists to patch over the larger problem of termination checking above.
  If a bisimilarity relation were to be introduced, it can be removed.
  \item Two free theorems.
  The postulating of the free theorems was needed as it is currently not possible to prove the correctness of free theorems from within Agda.
  New research does exist by \cite{Muylder2024} that would enable the proving of the two free theorems, using internalized parametricity as originally described by \cite{Bernardy2012}.
  Doing so falls outside the scope of this research and is left to future work.
\end{itemize}


My Agda formalization has shown that Harper's work is correct, with a couple limitations, namely with respect to the proof of terminality of $\nu$.
There are multiple of future avenues that this research can take, this is discussed more extensively in \autoref{sec:conclusion}.
