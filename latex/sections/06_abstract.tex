\thispagestyle{plain}
\begin{center}
    \Large
    \textbf{\thetitle}
        
    \vspace{0.4cm}
    \large
    \thesubtitle
        
    \vspace{0.4cm}
    \textbf{\thename}
       
    \vspace{1.9cm}
    \textbf{Abstract}
\end{center}
\begin{adjustwidth}{2cm}{2cm}
\textsc{Placeholder abstract pulled from ChatGPT:}
Functional programming's compositional techniques, such as those in Haskell, often incur computational overhead.
Fusion, an optimization method that combines multiple list operations into a single traversal, addresses this issue.
This thesis explores shortcut fusion using (Co)Church encodings, focusing on two key questions: how to reliably achieve fusion in Haskell, and the safety of these transformations.

The first contribution replicates and extends Harper's (Co)Church encodings in Haskell, uncovering optimizer weaknesses and practical fusion techniques.
The second formalizes these encodings in Agda, using category theory and containers, a generalization of strictly positive functors.
The formalization proves the equivalence of these encoded functions to standard folds, enhancing the understanding and applicability of fusion in functional programming.
\end{adjustwidth}

\vfill 

\begin{center}
    Thesis committee:\\~\\
    \begin{tabular}{r l l l}
        Chair: & Jesper Cockx & Programming Languages & Assistant Professor \\
        Core Member 2: & Casper Poulsen & Programming Languages & Assistant Professor \\
        Core Member 3: & Rihan Hai & Web Information Systems & Assistant Professor \\
        Daily Supervisor: & Jaro Reinders & Programming Languages & PhD Student 
    \end{tabular}
\end{center}
\vspace{1cm}
\pagebreak