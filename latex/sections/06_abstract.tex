\thispagestyle{plain}
\begin{center}
    \Large
    \textbf{\thetitle}
        
    \vspace{0.4cm}
    \large
    \thesubtitle
        
    \vspace{0.4cm}
    \textbf{\thename}
       
    \vspace{1.9cm}
    \textbf{Abstract}
\end{center}
\begin{adjustwidth}{2cm}{2cm}
When writing functional code that composes multiple recursive functions that operate on a datastrcuture, we often incur a lot of computational overhead allocating memory, only to later read, use, and discard this information.
This can be alleviated using fusion, a technique that combines these multiple recursive datastructure traversals into one.
This thesis explores shortcut fusion using (Co)Church encodings based on the work of \cite{Harper2011}, focusing on two questions: What is needed to reliably achieve fusion in Haskell, and the correctness of these transformations through a formalization in Agda.

The first contribution replicates and extends Harper's (Co)Church encodings in Haskell, uncovering optimizer weaknesses and providing practical insights for achieving fusion within Haskell.
The second contribution formalizes these encodings in Agda, leveraging parametricity and the category theory described by Harper.
The formalization proves the equivalence of these encoded functions to the unencoded ones, showing that the encodings are in fact isomorphisms, as long as parametricity \citep{Wadler1989} is assumed.

These findings highlight the effectiveness and correctness of shortcut fusion techniques and show the promise of shortcut fusion: Reduce the computational overhead associated with functional programming while retaining its nice, compositional properties.
\end{adjustwidth}

\vfill 

\begin{center}
    Thesis committee:\\~\\
    \begin{tabular}{r l l l}
        Chair: & Jesper Cockx & Programming Languages & Assistant Professor \\
        Core Member 2: & Casper Poulsen & Programming Languages & Assistant Professor \\
        Core Member 3: & Rihan Hai & Web Information Systems & Assistant Professor \\
        Daily Supervisor: & Jaro Reinders & Programming Languages & PhD Student 
    \end{tabular}
\end{center}
\vspace{1cm}
\pagebreak