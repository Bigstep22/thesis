\paragraph{Deriving a container from a functor}
Deriving the container from a given (polynomial) functor can be done in a few steps:
\begin{enumerate}
    \item Analyze how many constructors your functor has, take as an example 2.
    \item For the left side of the container take the coproduct of types that store the non-recursive sub-elements (such as const).
    \item Count the amount of recursive elements in the constructor, the return type should include that many elements.
\end{enumerate}
Taking an example:
\begin{itemize}
    \item[\tt{List}]
    Taking the base functor for \tt{List}: \tt{F$_A$ X := 1 + A $\times$ X}.

    For the \tt{Shape} we take the coproduct of \tt{Fin 1} and \tt{const A}, corresponding to the `\tt{nil}' and `\tt{cons a \_}' part, respectively.

    For the \tt{Position}, we have one constructor that is non-recursive and one that contains one recursive element, so we have:
    \tt{0 $\to$ Fin 0} and \tt{const n $\to$ Fin 1}.
    The Fin 1 refers to the recursive X that is present in the base functor (or the `\tt{cons \_ as}' part of cons).
    % Clean this up
    \item[\tt{Binary tree}]
    Taking the base functor for \tt{Tree}: \tt{F$_A$ X := 1 + X $\times$ A $\times$ X}.

    For the \tt{Shape} we take the coproduct of \tt{Fin 1} and \tt{const A}.

    For the \tt{Position}, we have one constructor that is non-recursive and one that contains two recursive elements, so we have:
    \tt{0 $\to$ Fin 0} and \tt{const n $\to$ Fin 2}.
    % Clean this up
\end{itemize}
We summarize the above \autoref{tab:cont_der}.
For a concrete example of how a datatype is implemented, see \autoref{sec:agda_church_list}.
\begin{table}[h]
  \centering
  \begin{tabular}{|r|l|l|}\hline
     & \tt{List} &  \tt{Binary Tree} \\\hline
    Base functor         & \tt{F$_A$ X := 1 + (A $\times$ X)}  & \tt{F$_A$ X := 1 + (X $\times$ A $\times$ X)} \\\hline
    Shape  & \tt{Fin 1 + const A} & \tt{Fin 1 + const A} \\\hline
    Position & \tt{nil $\to$ Fin 0 and const n $\to$ Fin 1} & \tt{nil $\to$ Fin 0 and const n $\to$ Fin 2} \\\hline
  \end{tabular}
  \caption{This table shows two examples of deriving the implementation of a container from a base functor.}
  \label{tab:cont_der}
\end{table}
An example of the implementation for Lists is discussed in \autoref{sec:agda_church_list}
